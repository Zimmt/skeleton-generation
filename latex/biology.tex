%---------------------------------
%---------------------------------
\chapter{Biologie der Wirbeltiere}
\label{chapter:biology}
% eventuell als Abschnitt in "`Grundlagen"'?

\begin{center}
 \begin{minipage}{12cm}
  \emph{"`Nichts anderes in der Natur hat eine herrlichere Struktur als der Körper der Wirbeltiere."'}
 
  --- "`Vergleichende und funktionelle Anatomie der Wirbeltiere"' \cite{Vergleichende_Anatomie}, S.\ 1
 \end{minipage}
\end{center}

Es folgt eine Einleitung zur biologischen Beschaffenheit der Wirbeltiere. Diese beschränkt sich aber auf eine kurze Definition und einige wichtige Merkmale und Eigenschaften, die zum Verständnis des Algorithmus wichtig sind.

% Definition
Wirbeltiere, oder auch Vertebrata, sind Tiere mit einer skelettartigen Schädelkapsel, einem Cranium. Deshalb werden sie auch Schädeltiere oder Craniota genannt.\\
Auf den ersten Blick ist diese Definition etwas überraschend, da als  Haupteigenschaft nicht die Wirbelsäule, sondern der Schädel genannt wird. Das liegt daran, dass zu den Wirbeltieren auch einige Tiere gezählt werden, die gar keine Wirbelsäule sondern "`nur"' eine Chorda dorsalis besitzen. Die Chorda dorsalis ist das Achsenskelett der Chordatiere, einem Tierstamm, zu dem auch die Wirbeltiere gehören. (\cite{Vergleichende_Anatomie}, S.27)\\ %und Wikipedia zu Wirbeltieren und chorda dorsalis
Hier in dieser Arbeit soll es aber nur um Wirbeltiere mit Wirbelsäule gehen.

% 5 Klassen
Es gibt fünf große Gruppen von Wirbeltieren. Die ersten bekannten Vertebraten sind die Fische. Aus ihnen entwickelten sich die Tetrapoden, Wirbeltiere mit vier Gliedmaßen. Zu den Tetrapoden gehören wiederum vier Klassen von Tieren. Zunächst gibt es die Amphibien, die noch nicht vollkommen terrestrisch leben. Daraus entwickelten sich die Reptilien, die erste Klasse die alle Strukturen besitzt um vollkommen an Land zu leben. Es gibt aber auch Reptilien die wieder im Wasser leben. Außerdem gibt es noch die Klasse der Vögel und der Säugetiere.

Das weithin bekannteste Säugetier ist der Mensch. Da sich Wirbeltiere, vor allem in ihrem Skelett, sehr ähneln, lassen sich viele der im Folgenden aufgeführten Eigenschaften vermutlich leicht "`am eigenen Leib"' nachvollziehen.


%------------------------------------
\section{Das Skelett}
\label{biology_skeleton}

\vspace{0.5cm}
\begin{center}
 \begin{minipage}{12cm}
  \emph{"`Das innere, gelenkige Skelettsystem der Vertebraten ist einzigartig im Tierreich. Es ist das wichtigste aller Organsysteme für das Studium der Wirbeltiermorphologie."'}
 
  --- "`Vergleichende und funktionelle Anatomie der Wirbeltiere"' \cite{Vergleichende_Anatomie}, S.\ 131
 \end{minipage}
\end{center}

Wirbeltiere haben ein inneres Skelett und ihr Körper ist bilateralsymmetrisch (ihre linke Körperseite ist symmetrisch zur rechten). 

% Wirbelsäule
Die \emph{Wirbelsäule} ist je nach Lebensweise des entsprechenden Wirbeltiers aufgebaut. Ein wesentlicher Einflussfaktor sind die Kräfte, die auf die Wirbelsäule wirken. 
Bei Fischen muss die Wirbelsäule dem Druck der starken Axialmuskeln, die seitlich am Körper entlang führen, entgegenwirken.
Bei Tetrapoden sind diese Axialmuskeln zurückgebildet, dafür muss die Wirbelsäule aber der Schwerkraft wiederstehen.
Bei Vögeln ist die Wirbelsäule noch einmal mehr spezialisiert, da sie an den Flug angepasst werden muss.

% Wirbelanzahl
Auch die Anzahl der Wirbel unterscheidet sich teilweise erheblich. 
Die Halswirbelsäule ist \zb bei den Fischen noch gar nicht herausgebildet. Amphibien haben einen Wirbel, der für die Beweglichkeit des Kopfes zuständig ist. Bei Reptilien ist die Halsregion meist schon stärker abgesetzt. Säugetiere haben, bis auf wenige Ausnahmen, genau $7$ Halswirbel, egal wie lang der Hals ist. Vögel haben die meisten Halswirbel, nämlich von $10$ bis $31$ beim Trauerschwan \cite{WikipediaVogelskelett}, meistens aber zwischen $15$ und $20$. (siehe auch \cite{Vergleichende_Anatomie}, Abschnitt 9.2, S.\ 168 ff.)

Ebenso unterscheidet sich die Anzahl der Wirbel auf anderen Teilen der Wirbelsäule. Teilweise sind Wirbel sogar miteinander zu einem größeren Knochen verwachsen. Betrachtet man nur die Säugetiere, so kommt man auf auf etwa $12$ bis $14$ Brustwirbel, $5$ bis $7$ Lendenwirbel, $3$ bis $5$ Kreuzwirbel und $4$ bis $22$ Schwanzwirbel \cite{AnatomieKuenstler}.

% Wirbelform
Auch die Form der einzelnen Wirbel unterscheidet sich sowohl zwischen den verschiedenen Tieren als auch entlang der Wirbelsäule einer einzelnen Art (\cite{Vergleichende_Anatomie}, Abschnitt 9.1 und Abbildung 9.2). Diese Vielfalt in ein Regelwerk zu zwängen, um sie automatisch generieren zu können, erscheint aussichtslos. Deshalb wird hier nicht weiter darauf eingegangen.

% Extremitäten
Die \emph{Extremitäten} der Tetrapoden sind nach einem "`Grundbauplan"' aufgebaut (siehe Abbildung \ref{grundbauplan}). Dieser wird jedoch vielfach abgewandelt. Einige Beispiele sind in den Abbildungen \ref{bsp_extremitaeten1} und \ref{bsp_extremitaeten2} zu sehen. (\cite{AllgemeineZoologie}, S.\ 487)\\
Bei Fischen haben die Extremitäten keine Verbindung zur Wirbelsäule. Erst Amphibien bilden hier eine Verbindung, da sie zur Fortbewegung an Land nötig ist. (\cite{Vergleichende_Anatomie}, Abschnitt 9.2.3)\\
Ein Wirbeltierskelett ist also nicht unbedingt zusammenhängend.

\begin{figure}
 \centering
 \includegraphics[width=0.8\textwidth]{graphics/GrundbauplanExtremitaet.pdf}
 \caption{\emph{Grundbauplan} der Extremitäten von Tetrapoden: Humerus = Oberarmknochen, Femur = Oberschenkelknochen, Radius = Speiche, Ulna = Elle, Tibia = Schienbein, Fibula = Wadenbein, Metacarpus = Mittelhand, Metatarsus = Mittelfuß. (\cite{AllgemeineZoologie}, S.\ 487; eine ähnliche Abbildung ist auch in \cite{Vergleichende_Anatomie}, S.\ 184, zu finden)}
 \label{grundbauplan}
\end{figure}

\begin{figure}
 \centering
 \includegraphics[width=0.8\textwidth]{graphics/ExtremitaetenBeispiele.pdf}
 \caption{Skelett der linken vorderen Extremität bei Huftieren. \textbf{a} Tapir, \textbf{b} Rhinoceros, \textbf{c} Pferd, \textbf{d} Schwein, \textbf{e} Rind, \textbf{f} Kamel. Unterarm weiß, Handwurzel schwarz, Mittelhand weiß, Finger schwarz. \textbf{R} Radius, \textbf{U} Ulna, \textbf{2-5} Mittelhandknochen, \textbf{II-V} Finger. (\cite{AllgemeineZoologie}, S.\ 487)}
 \label{bsp_extremitaeten1}
\end{figure}

\begin{figure}
 \centering
 \includegraphics[width=0.8\textwidth]{graphics/ExtremitaetenBeispiele2.pdf}
 \caption{Vordergliedmaßen von Mensch (1), Eidechse (2), Wal (3), Maulwurf (4), Pinguin (5), Pferd (6), Flugsaurier (7), Vogel (8) und Fledermaus (9). (\cite{dtvBiologie}, S.\ 474)}
 \label{bsp_extremitaeten2}
\end{figure}

%Gelenke
Auch bei \emph{Gelenken} gibt es viele verschiedene Ausprägungen (siehe \cite{Vergleichende_Anatomie}, Absatz 21.5.2). Der Einfachheit halber werden hier nur drei Arten von Gelenken betrachtet: unbewegliche Gelenke (\zb zwischen Wirbeln), Gelenke mit einem Freiheitsgrad (\zb an Ellenbogen oder Knie) und Gelenke mit zwei Freiheitsgraden (\zb an Schulter oder Hüfte).
\todo{Illustration der Gelenke in Extremitäten}\\
Außerdem wird angenommen, dass die jeweiligen Bewegungsradien für alle Tiere gleich sind.

% Extremitätengürtel
Schulter- und Beckengürtel bestehen  jeweils aus mehreren Knochen, die je nach Art mehr oder weniger ausgebildet, unterschiedlich geformt oder sogar komplett zurückgebildet sein können. Um das im Folgenden zu vereinfachen, wird der Schultergürtel durch ein Schulterblatt repräsentiert und der Beckengürtel duch einen "`Beckenknochen"', der allerdings eigentlich auch wieder aus mehreren Einzelknochen besteht.

\todo{Schematischer Aufbau eines Wirbeltierskeletts, Reduktion auf "`wesentliche"' Knochen}


%-----------------------------------------------------
\section{Fun Facts (die später noch gebraucht werden)}

\begin{itemize}
 \item schwerstes Tier
 \item Ähnlichkeiten und Unterschiede (\zb große vs. kleine Tiere, kann man sie einfach skalieren?; Knochen die sich stark oder weniger stark unterschieden)
\end{itemize}

%-------------------
\section{Mythologie}

\begin{itemize}
 \item Zentauren haben Körper wie ein Pferd, aber statt einem Hals setzt ein menschlicher Oberkörper auf Schultergürtel auf.\\
 \url{https://tvtropes.org/pmwiki/pmwiki.php/Main/OurCentaursAreDifferent?from=Main.CentauroidForm}
 \item Pegasus hat den Körper eines Pferdes + Flügel, die zusätzlich zu den Vorderbeinen an Schultergürtel ansetzen\\
 \url{https://tvtropes.org/pmwiki/pmwiki.php/Main/Pegasus}
\end{itemize}

