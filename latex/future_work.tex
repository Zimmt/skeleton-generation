%---------------------------
%---------------------------
\chapter{Fazit und Ausblick}
\label{chapter:conclusion}

% 1. Zusammenfassung: Was wurde in dieser Arbeit gemacht? Was waren die Schlüsselergebnisse? Diesmal jedoch zusammengefasst nochmals für einen Leser, der die vorherigen Seiten mit allen Details bereits gelesen hat.
% 
% 2. Adressaten der Verbesserung: Wem nützen die Verbesserungen/Beiträge der Arbeit? Inwieweit wird die Software-Technik durch die Arbeit verbessert?
% 
% 3. Aufbauende/Zukünftige Arbeiten: Welche nächsten Schritte sind geplant (erst kurzfristige, dann längerfristige)? Welche möglichen Lösungsansätze für noch bestehende Probleme sind denkbar? Wie könnten Folgearbeiten aussehen?

\paragraph{Zusammenfassung}

Das Ergebnis dieser Arbeit ist ein Algorithmus, der 3D-Modelle von Wirbeltierskeletten generieren kann, und eine Implementierung desselben. Die Informationen, auf deren Grundlage der Algorithmus arbeitet, sind annotierte 2D-Bilder von Wirbeltierskeletten, \zb aus Lehrbüchern der Zoologie. Mit Hilfe einer \emph{Principal Component Analysis} auf diesen Datenpunkten (siehe Kapitel \ref{chapter:pca}) erhält man eine Möglichkeit zufällig normalverteilte Punkte zu erzeugen, die der gleichen Verteilung folgen wie die Datenpunkte.
Ein so erzeugter Punkt liefert alle Informationen, die nötig sind, um ein Skelett zu generieren.
Diese Technik liefert ohne weitere Einschränkungen viele unterschiedliche Skelette, die in den allermeisten Fällen auch gut aussehen.\\
Außerdem können gezielt Skelette mit bestimmten Eigenschaften generiert werden, indem die \emph{Principal Component Analysis} auf bedingten Verteilungen ausgeführt wird oder Variationen zu schon generierten Skeletten erzeugt werden. Dies ist sehr hilfreich, wenn man schon eine Vorstellung davon hat, wie das zu generierende Skelett ungefähr aussehen soll.

Zusätzlich zu "`realistischen"' Wirbeltierskeletten lassen sich mit dem Algorithmus auch fantastische Skelette generieren. Es können zusätzliche Extremitäten und ein zusätzlicher Schultergürtel erzeugt werden.
Die Datengrundlage bleibt jedoch die selbe. Es werden die gleichen Bilder von echten Wirbeltierskeletten verwendet.
Deshalb wurde darauf geachtet, dass sich die generierten Skelette durch die genannten Veränderungen nicht zu stark von echten Wirbeltierskeletten entfernen.  

Eine kontextfreie Grammatik legt fest, welche Knochen generiert werden und wie die Baumstruktur des Skeletts aufgebaut ist.
Diese Grammatik sorgt dafür, dass die Ersetzungsregeln relativ übersichtlich dargestellt und leicht erweiterbar sind. Da sie aber im Voraus, durch den zufällig erzeugten Punkt, schon so stark festgelegt wird, dass immer genau ein Wort generiert wird, ist sie nicht unbedingt notwendig.

Die Positionierung der Beine liefert in vielen Fällen keine schönen Ergebnisse, da hierzu keine Informationen aus den 2D-Bildern erhoben werden. Trotzdem kann das generierte Skelett als erster Eindruck \bzw Inspiration für ein Modell dienen. Wenn das erzeugte Skelett die Grundlage für ein animiertes Wesen sein soll, muss die Positionierung der Knochen sowieso noch öfter angepasst werden.


\paragraph{Zukünftige Arbeiten}





\begin{itemize}
 \item schnell viele verschiedene Skelette generierbar, Variationen zu einem vorgegebenen Skelett, leicht bedienbar

 \item mehr Daten für PCA sammeln
 \item Erweiterung auf Mensch? Mensch auch als PCA Datenpunkt?
 \item austauschbare 3D-Modelle,  mehr verschiedene 3D-Modelle
 \item Gewicht in Algo verwenden
 
 \item Interaktivität: Teile, die einem nicht gefallen, sollten geändert werden können (Eine Anwendung, bei der nach Eingabe von Parametern sofort das komplette Tier generiert wird, ist weniger hilfreich als eine, bei der schrittweise Teile davon generiert werden können (und auch rückgängig gemacht werden können))
 
 \item Muskeln, Haut \etc anbauen (an 3D-Modell oder auch Algo erweitern)
\end{itemize}


