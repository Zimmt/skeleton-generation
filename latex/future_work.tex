%-----------------------------------------------------
%-----------------------------------------------------
\chapter{Fazit und Ausblick}
\label{chapter:conclusion}

\begin{itemize}
 \item Muskeln, Haut \etc anbauen
 \item austauschbare 3D-Modelle
 \item mehr verschiedene 3D-Modelle
 \item Beinalgo verbessern / IK
 \item mehr Daten für PCA sammeln
 \item Gewicht in Algo verwenden
 \item Erweiterung auf Mensch? Mensch auch als PCA Datenpunkt?
 \item Interaktivität: Teile, die einem nicht gefallen, sollten geändert werden können
 \item Eine Anwendung, bei der nach Eingabe von Parametern sofort das komplette Tier generiert wird, ist weniger hilfreich als eine, bei der schrittweise Teile davon generiert werden können (und auch rückgängig gemacht werden können)
\end{itemize}


%----------------
\section{Muskeln}

\begin{itemize}
 \item Die "`Hauptmuskeln"' verlaufen bei Wirbeltieren wahrscheinlich ähnlich, relativ zu den Knochen. Trotzdem unterscheiden sie sich recht stark.
 \item Knochen/Gelenke bekommen Zusatzattribute für Start- und Zielpunkte der Muskeln.
 \item Muskeln haben eine "`Dicke"' und aus Start- und Zielpunkt muss Kurve des Muskels generiert werden.
 \item Wie wird die genaue Form festgelegt? Muskeln irgendwie auf ihre "`Dicke"' aufblähen + Interaktion mit vorhandenen Elementen (andere Muskeln und Knochen) $\rightarrow$ future work
\end{itemize}

%-------------
\section{Haut}

\begin{itemize}
 \item Was für Algorithmen gibt es, die zu einem vorhandenen 3D-Modell eine Hülle mit gewissen Eigenschaften generieren? \\
 es gibt eine solche Funktion z.B. in AutoCAD \url{https://knowledge.autodesk.com/de/support/autocad/learn-explore/caas/CloudHelp/cloudhelp/2016/DEU/AutoCAD-Core/files/GUID-B7F99810-765E-4E7E-ABDD-275C64147CCC-htm.html}
 \item Einfach nur eine Hülle mit gewissem Abstand sieht wahrscheinlich sehr unrealistisch aus. "`Bony Landmarks"' (Stellen an denen das Gewebe über den Knochen sehr dünn ist) könnten helfen (siehe \url{https://www.proko.com/landmarks-of-the-human-body/}) oder "`bone weights"'
\end{itemize}
