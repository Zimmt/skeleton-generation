%---------------------------
%---------------------------
\chapter{Fazit und Ausblick}
\label{chapter:conclusion}

% 1. Zusammenfassung: Was wurde in dieser Arbeit gemacht? Was waren die Schlüsselergebnisse? Diesmal jedoch zusammengefasst nochmals für einen Leser, der die vorherigen Seiten mit allen Details bereits gelesen hat.
% 
% 2. Adressaten der Verbesserung: Wem nützen die Verbesserungen/Beiträge der Arbeit? Inwieweit wird die Software-Technik durch die Arbeit verbessert?
% 
% 3. Aufbauende/Zukünftige Arbeiten: Welche nächsten Schritte sind geplant (erst kurzfristige, dann längerfristige)? Welche möglichen Lösungsansätze für noch bestehende Probleme sind denkbar? Wie könnten Folgearbeiten aussehen?

Das Ergebnis dieser Arbeit ist ein Algorithmus, der 3D-Modelle von Wirbeltierskeletten generiert, und eine Implementierung desselben. Die Daten, auf deren Grundlage der Algorithmus arbeitet, sind annotierte 2D-Bilder von Wirbeltierskeletten, \zb aus Lehrbüchern der Zoologie. 
Mit Hilfe einer \emph{Principal Component Analysis} (PCA) erhält man die Möglichkeit zufällig normalverteilte Punkte zu erzeugen, die der gleichen Verteilung folgen wie die Beispieldaten.
Ein so erzeugter Punkt liefert alle benötigten Informationen um ein Skelett zu generieren.
Diese Technik liefert ohne weitere Einschränkungen viele unterschiedliche Skelette, die in den allermeisten Fällen auch realistisch wirken.\\
Nur die Positionierung der Beine liefert in vielen Fällen keine schönen Ergebnisse. Um das zu verbessern wären weitere Informationen, \zb aus den 2D-Bildern, notwendig. Trotzdem kann das generierte Skelett als erster Eindruck \bzw Inspiration für ein Modell dienen. Wenn das erzeugte Skelett die Grundlage für ein animiertes Wesen sein soll, muss die Positionierung der Knochen ohnehin situationsgerecht angepasst werden.

Außerdem können gezielt Skelette mit bestimmten Eigenschaften generiert werden, indem PCA auf bedingten Verteilungen ausgeführt wird oder Variationen zu schon generierten Skeletten erzeugt werden. Dies ist sehr hilfreich, wenn man schon eine Vorstellung davon hat, wie das zu generierende Skelett ungefähr aussehen soll.

Eine kontextfreie Grammatik legt fest, welche Knochen generiert werden und wie das Skelett aufgebaut ist.
Durch zusätzliche Bedingungen wird sie so eingeschränkt, dass sie immer genau ein Skelett generiert. Man könnte die Vorschriften, die in der Grammatik codiert sind, also stattdessen auch als einfachen Algorithmus formulieren.
Die Darstellung als Grammatik sorgt jedoch dafür, dass die Ersetzungsregeln relativ übersichtlich und leicht erweiterbar sind. 

Zusätzlich zu "`realistischen"' Wirbeltierskeletten lassen sich mit dem Algorithmus auch fantastische Skelette generieren. Es können zusätzliche Extremitäten und ein zusätzlicher Schultergürtel erzeugt werden, ohne dass die Datengrundlage verändert werden muss. Es werden die gleichen Bilder von echten Wirbeltierskeletten verwendet.
Deshalb unterscheiden sich die fantastischen Skelette nicht zu sehr von echten Wirbeltierskeletten.  


\paragraph{Ausblick}

In zukünftigen Arbeiten könnte es sinnvoll sein die Datengrundlage, also die Auswahl an 2D-Skelettbildern, zu erweitern. Das Beispiel Mensch wurde \zb außen vor gelassen. Es könnte also getestet werden wie gut Skelette mit sehr aufrechter Wirbelsäule generiert werden können.\\
Außerdem könnte das Merkmal \emph{Gewicht}, das hier zwar erhoben, aber nicht verwendet wurde, in die Generierung mit einfließen, \zb in die Dicke der Knochen. Hier sollte aber beachtet werden, dass die Knochendicke der Tiere nicht proportional zu ihrem Gewicht ist (siehe Abschnitt \ref{bigAndSmall}).

Die Sammlung der verwendeten 3D-Modelle von Knochen lässt sich relativ leicht durch weitere \bzw alternative Modelle erweitern (siehe Abschnitt \ref{bone_models}). Um die Benutzung des Programms noch flexibler zu machen, könnte es so ergänzt werden, dass die zu verwendenden Knochenmodelle vom Benutzer selbst vorgegeben werden können.\\
Desweiteren könnte der Algorithmus auch interaktiv gestaltet werden, um dem Benutzer noch mehr Kontrolle über die Generierung zu geben. Diese könnte \zb schrittweise erfolgen, mit der Möglichkeit nach jedem Schritt einzugreifen. Der Benutzer könnte dabei beispielsweise Schritte rückgängig machen, Skelettteile, die ihm nicht gefallen, neu generieren oder zusätzliche Bedingungen angeben.

Aufgrund der Zeitvorgabe beschränkt sich diese Arbeit auf die Generierung von Skeletten. Es werden weder Muskeln noch Haut generiert. Es wäre aber möglich den hier vorgestellten Algorithmus entsprechend zu erweitern. Dies würde einen noch plastischeren Eindruck der generierten Tiere liefern und die Modellierung von sehr wirklichkeitsnahen Wirbeltieren noch mehr vereinfachen.


