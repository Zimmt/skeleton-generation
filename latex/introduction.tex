%-------------------
%-------------------
\chapter{Einleitung}

Wirbeltiere sind allgegenwärtig. Nicht nur im alltäglichen Leben, sondern auch in Computerspielen und Filmen. Es besteht also ein stetiger Bedarf an 3D-Modellen von Wirbeltieren. Dieser wird in den meisten Fällen durch KünstlerInnen gedeckt, die viel Zeit und Arbeit in die Modelle stecken. 
Algorithmen, die diese Abläufe vereinfachen könnten, beispielsweise durch prozedurale Generierung, gibt es nicht viele \cite{PCGSurvey_videoGames}.\\
Die meisten Ansätze, die es gibt, generieren 3D-Modelle mit einem sehr abstrakten Skelett (Rig) für die Animation (siehe \mbox{Abschnitt \ref{procedural_generation}}). 
Der Algorithmus, der in dieser Arbeit vorgestellt wird, konzentriert sich jedoch nur auf das Skelett. Es werden abstrahierte, aber dennoch recht wirklichkeitsnahe, Modelle generiert. Diese können dann die Grundlage für die Modellierung eines kompletten Tieres bilden.

% Ziel
Einerseits können die generierten Skelette als Inspiration dienen. Sie können einen Anhaltspunkt zu Körperform, Proportionen und zu möglichen Bewegungsabläufen bieten und es kann schnell eine große Auswahl von ihnen generiert werden.
Andererseits können die Skelette eine Grundlage für Modelle sein, die zusätzlich auch Muskeln und Haut simulieren.
In beiden Fällen darf das Modell nicht zu abstrakt sein, da es sonst zu wenig Informationen zum Aufbau des konkreten Tiers liefert.\\
Es soll also ein Skelett generiert werden, das genug Knochen enthält um realistisch zu wirken, aber auch nicht zu viele um den Aufwand für die Generierung und die Programmierung des Algorithmus im Rahmen zu halten.

% Warum nur Wirbeliere?
Diese Arbeit beschränkt sich auf Wirbeltierskelette, da ihr Aufbau nicht sehr variiert. 
Je mehr Details betrachtet werden, desto mehr Unterschiede treten hervor. Der grobe Aufbau ist aber bei allen Wirbeltieren sehr ähnlich. Die Grundlagen zur Biologie der Wirbeltiere, die für diese Arbeit notwendig sind, werden in Kapitel \ref{chapter:biology} vorgestellt, die technischen Grundlagen in Kapitel \ref{chapter:basics}.

% Methoden+Aufbau
Die wichtigsten Schritte des Algorithmus werden im Folgenden kurz angerissen, um einen Überblick über die Arbeit zu geben. Ein Überblick über den Algorithmus ist auch in Abschnitt \ref{section:overview} zu finden.\\
Die Datengrundlage für den Algorithmus schafft eine \emph{Principal Component Analysis} (Hauptkomponentenanalyse) auf annotierten 2D-Skelett"-bildern (Kapitel \ref{chapter:pca}). Ohne sie wäre es schwer Skelette in einer natürlichen und stabilen Haltung zu generieren.
Die Knochen des Skeletts werden dann mit Hilfe einer kontextfreien Grammatik generiert und gleichzeitig angeordnet (Kapitel \ref{chapter:skeleton_generation}). 
Zum Schluss wird aus existierenden 3D-Modellen von einzelnen Knochen ein 3D-Modell des generierten Skeletts erstellt (Abschnitt \ref{bone_models})).

Prinzipiell generiert der Algorithmus zufällige Skelette. Es können aber auch Benutzereingaben, \zb zur Anzahl der Extremitäten, berücksichtigt werden oder Variationen zu schon bestehenden Skeletten generiert werden (Kapitel \ref{chapter:additional_features}). \\
Zusätzliche Informationen zu Implementierungsdetails sind in Kapitel \ref{chapter:implementation_detail} zu finden und abgerundet wird die Arbeit mit Fazit und Ausblick in Kapitel \ref{chapter:conclusion}.



