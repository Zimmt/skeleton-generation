%-------------------
%-------------------
\chapter{Einleitung}


%----------------------
%----------------------
\chapter{Idee und Ziel}

\begin{itemize}
 \item Theorie + Implementierung
 \item prozedurale Generierung
 \item Es gibt schon Pflanzen, Landschaften, Wolken \etc aber keine Tiere
 \item viele verschiedene (und möglicherweise doch ähnliche) Tiere generieren für Spiel/Film
 \item Inspiration für Künstler, 3D-Animation
    \begin{itemize}
     \item Skelett darf nicht zu abstrakt sein, da es sonst zu wenig Informationen zum konkreten Tier liefert.
     \item Ein detailliertes Skelett ist für Wesen sinnvoll, die es so noch nicht gibt bzw. wo man keine richtige Vorstellung davon hat wie es "`funktioniert"', z.B. bei mehr Gliedmaßen.
    \end{itemize}
 \item Einschränkung auf Skelett, da sonst zu umfangreich für eine Masterarbeit
 \item Wirbeliere, da Skelette sehr ähnliche
 \item möglicherweise interaktive Bedienung
\end{itemize}

\begin{enumerate}
 \item Erzeugung eines Wirbeltierskeletts
 \item Erzeugung von Muskeln (future work)
 \item Erzeugung von Haut (future work / kurz ausprobieren)
 \item Erzeugung von vielen Skelettvarianten bei Eingabe eines Skeletts (nur wenn es relativ leicht möglich ist)
\end{enumerate}

Interaktivität

\begin{itemize}
 \item Eine Anwendung, bei der nach Eingabe von Parametern sofort das komplette Tier generiert wird, ist weniger hilfreich als eine, bei der schrittweise Teile davon generiert werden können (und auch rückgängig gemacht werden können)
 \item Teile, die einem nicht gefallen, sollten geändert werden können
 \item Änderungen können Auswirkungen auf den Rest des Körpers haben (durch Regeln) bzw. manche Änderungen sind nicht möglich
 \item Könnte verwendet werden um schnell verschiedene Möglichkeiten zu testen
\end{itemize}



