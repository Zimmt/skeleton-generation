%-------------------------------------------
%-------------------------------------------
\chapter{Algorithmus zur Skelettgenerierung}

%-----------------------------------------------
\section{Generierung der einzelnen Extremitäten}
\label{section:extremity_generation}

Extremitäten werden zunächst danach unterschieden, ob sie Bodenkontakt haben sollen oder nicht.
Wenn nicht, dann sind es entweder Flossen, Arme oder Flügel:

\begin{itemize}
 \item Flossen: gerade nach hinten (orientiert an Welt-x-Achse)
 \item Arme: Oberarm gerade nach unten (orientiert an negativer Welt-y-Achse), Unterarm im $90^{\circ}$ Winkel nach vorne, Hand verlängert Unterarm 
 \item Flügel: spezielles Winkelintervall für jedes beteiligte Gelenk, daraus jeweils zufällig gewählte Winkel
\end{itemize}

Bei Extremitäten mit Bodenkontakt wird iterativ vorgegangen. \todo{Der allgemeinste Ansatz wäre hier inverse Kinematik zu verwenden. Das ist hier aber nicht nötig, da in jedem Schritt klar ist, wie die Winkel verändert werden müssen, dass der Endpunkt näher zum Boden kommt.}
Von der Startposition aus, werden die Winkel an den Gelenken in jedem Schritt jeweils so vergrößert oder verkleinert, dass sich der Punkt, der zum Schluss den Boden berühren soll, sich der Bodenoberfläche nähert. Ob die Winkel jeweils vergößert oder verkleinert werden sollen, wird bestimmt, indem die Ausrichtung des Kindelements mit der y-Achse des Weltkoordinatensystems verglichen wird. Soll der Endpunkt der Extremität dem Boden nähern, so wird der Winkel so verändert, dass die Ausrichtung des Kindelements sich der Senkrechten nähert. Wenn nicht, so wird der Winkel in die entgegengesetzte Richtung verändert. \todo{das passiert aber nicht}

Je nach Ausgangsposition sieht das Ergebnis aber nicht unbedingt natürlich aus. Zum Beispiel kann es passieren, dass das Fußgelenk nicht gedreht wird, also der Fuß das Schienbein einfach verlängert und die Spitze des Fußes Bodenkontakt hat. Wenn dann die Oberseite des Fußes näher am Boden ist als die Unterseite, dann ist das keine sinnvolle Position.
Um so etwas zu verhindern, wird die Startposition der Extremität so gewählt, dass alle Gelenke stark angewinkelt sind. \todo{Abbildung}
Außerdem wird während der Iteration verboten, dass Knochen unterhalb der Bodenhöhe enden.

In jedem Schritt werden die Winkel um eine bestimmte Gradzahl verändert. Diese Gradzahl verkleinert sich mit jedem Schritt bis zu einer Minimalgröße. Zu Beginn werden die Winkel stark verändert um die grobe Ausrichtung des Beines festzulegen und in den kleiner werdenen Schritten wird die Extremität genauer ausgerichtet, so dass der Endpunkt zum Schluss auf dem Boden steht.
Zusätzlich wird nicht in jedem Schritt jeder Freiheitsgrad jedes Gelenks verändert. Für jeden Freiheitsgrad wird eine Wahrscheinlichkeit (kleiner als eins) festgelegt, dass dieser ausgewählt wird. Dadruch können bestimmte Richtungen oder Gelenke priorisiert werden um ein besseres Ergebnis zu erzielen. \todo{Was sind die guten Einstellungen? bzw braucht man dise Wkten überhaupt?}
\todo{Konkrete Einstellungen erwähnen (in Implementierungsdetails?)}

%--------------------------------------
\section{Ansatzpunkte für Extremitäten}

Ansatzpunkte für Extremitäten sind natürlich zunächst der Hüftgürtel und der Schultergürtel. Um auch die Generierung fantastischer Tiere zu ermöglichen, ist es aber Möglich dies zu erweitern.

Eine einfache Möglichkeit ist hier zunächst die Anzahl der möglichen Extremitätenpaare von zwei auf vier zu erhöhen, indem einfach an der Hüfte und der Schulter jeweils zwei Paare ansetzen dürfen. Dafür wurden einfach an der Hüfte \bzw der Schulter mehrere Gelenke mit ein wenig Abstand angelegt, an denen Extremitäten ansetzen können.
Flügel und Arme dürfen hierbei weiterhin nur an der Schulter ansetzen, Beine und Flossen an beiden Stellen. Der Grund dafür ist, dass die meisten generierten Skelette seltsam wirken, wenn an der Hüfte Flügel oder Arme ansetzen und dafür an der Schulter Beine beginnen. Das liegt daran, dass existierende Tiere mit Flügeln oder Armen ihren Schwerpunkt im hinteren Bereich haben und sie auf den Hinterbeinen stehen.

% mehr Extremitätengürtel auf dem Rücken
Eine Überlegung war auch zwischen Schulter und Hüfte weitere Extremitätengürtel anzubringen. Das stellt sich aber als schwierig heraus. Die Wirbelsäule ist zwischen Hüfte und Schulter nach oben geschwungen und im Bauchraum befinden sich die meisten Organe des Tieres. Ein zusätzlicher Extremitätengürtel würde den Bauchraum einschränken. Außerdem wirkt dann auch die nach oben geschwungene Wirbelsäule anatomisch seltsam.
"`Verdoppelt"' man die Schwingung der Wirbelsäule und hängt einfach einen weiteren Rücken hinten oder vorne an, so wirkt es ebenso seltsam, da dann die "`Höcker"' der Wirbelsäule für das Tier wahrscheinlich nicht wirklich ein Vorteil sind und nur die Fortbewegung erschweren.

% zweiter Schultergürtel
Eine weitere Idee, die auch umgesetzt wurde, ist, eine Art Zentauren zu ermöglichen. Hat das Tier einen Hals, der lang genug ist, kann darauf ein weiterer Schultergürtel kurz unterhalb vom Kopf angebracht werden. An diesem Schultergürtel dürfen dann keine alle Arten von Extremitäten außer Beinen ansetzen. Das wirkt tatsächlich meist auch anatomisch einigermaßen sinnvoll.


%--------------------------------
\section{Knochenmodelle einfügen}

Zunächst wird jeder terminale Knochen durch seine Bounding Box dargestellt. Diese Boxen lassen sich aber leicht durch die 3D-Modelle der entsprechenden Knochen ersetzen. Dazu müssen die 3D-Modelle nur im .obj-Format vorliegen und folgenden Bedingungen entsprechen: \\
Das Modell ist korrekt an den Achsen ausgerichtet und so verzerrt, dass es einen Würfel mit 1(m) Kantenlänge in jeder Richtung möglichst gut ausfüllt.
Es ist auch möglich kleine Fortsätze (wie \zb diejenigen von Wirbeln) über den Würfel hinaus stehen zu lassen. Das macht unter Umständen die Positionierung einfacher.
Bei der Hüfte wurde das Modell beispielsweise so ausgerichtet, dass Anfangs- und Endpunkt der durchgehenden Wirbelsäule jeweils auf dem Rand des Würfels liegen. Das vereinfacht die Positionierung anschließender Wirbel erheblich.

Liegen die Modelle in diesem Format vor, können sie einfach eingelesen werden und anhand der Skalierung der Bounding Box skaliert werden. 

% Ausrichtung der Knochen
Eine Schwierigkeit daran Modelle in der oben genannten Form herzustellen ist, dass nicht unbedingt sofort klar ist, wie die Knochen ausgerichtet werden müssen. Die Hüfte muss \zb so ausgerichtet werden, dass der Anfangs- und Endpunkt der durchgehenden Wirbelsäule auf gleicher Höhe liegen, damit nachfolgende Wirbel auch richtig anschließen. (Das funktioniert natürlich nur, weil die Wirbelsäule an der Stelle der Hüfte quasi gerade ist.)
Auch die Positionierung der Rippen und des Oberarms in Kombination mit dem Unterarm ist anspruchsvoll. \todo{Beschreiben wenn zufrieden mit Positionierung}

% Gelenke korrekt ausrichten
Zusätzlich muss beachtet werden wie die Knochen aneinander anschließen \bzw wie sie für die entsprechenden Gelenke korrekt positioniert sind. Das erfordert etwas "`finetuning"'. Für jeden Knochen sind dafür in Abhängigkeit zur Bounding Box zwei Offsets gespeichert: das Offset zu dem Gelenk, das ihn mit seinem Elternknochen verbindet und das Offset zu dem Gelenk, das ihn mit seinem Kindknochen verbindet (oder mehrere, falls vorhanden). Dies sorgt dafür, dass die Positionierung der Knochen stimmt, egal wie groß sie sind. Ist ein Knochen sehr groß und ein anschließender sehr klein (oder anders herum), so kommt es natürlich trotzdem vor, dass die Gelenke nicht wirklich ineinander passen. Für solche Situationen bräuchte man verschiedene 3D-Modelle, die je nach Gegebenheit eingesetzt werden.



