%-------------------------------------------
%-------------------------------------------
\chapter{Algorithmus zur Skelettgenerierung}

%-----------------------------------------------
\section{Generierung der einzelnen Extremitäten}
\label{section:extremity_generation}

Extremitäten werden zunächst danach unterschieden, ob sie Bodenkontakt haben sollen oder nicht.
Wenn nicht, dann sind es entweder Flossen, Arme oder Flügel:

\begin{itemize}
 \item Flossen: gerade nach hinten (orientiert an Welt-x-Achse)
 \item Arme: Oberarm gerade nach unten (orientiert an negativer Welt-y-Achse), Unterarm im $90^{\circ}$ Winkel nach vorne, Hand verlängert Unterarm 
 \item Flügel: spezielles Winkelintervall für jedes beteiligte Gelenk, daraus jeweils zufällig gewählte Winkel
\end{itemize}

% Beine mit iterativem Algo
Bei Extremitäten mit Bodenkontakt wird iterativ vorgegangen. Der allgemeinste Ansatz wäre hier inverse Kinematik zu verwenden. Das ist hier aber nicht nötig, da in jedem Schritt klar ist, wie die Winkel verändert werden müssen, dass der Endpunkt näher zum Boden kommt. \todo{Absatz über IK, lcp (linear complementary problem) (nicht Hauptaugenmerk / Ziel der Arbeit ist etwas anderes / reicht für Proof of Concept, bei Animationen muss große Maschinerie sowieso nochmal angeworfen werden)}\\
\todo{Absatz über Darstellung der Gelenke (auch mit zwei Freiheitsgraden), Problematik mit lokalen Winkelkonstraints vs. globalen Berechnungen für Abstand zum Boden}
Von der Startposition aus, werden die Winkel an den Gelenken in jedem Schritt jeweils so vergrößert oder verkleinert, dass sich der Punkt, der zum Schluss den Boden berühren soll, sich der Bodenoberfläche nähert. Ob die Winkel jeweils vergößert oder verkleinert werden sollen, wird bestimmt, indem die Ausrichtung des Kindelements mit der y-Achse des Weltkoordinatensystems verglichen wird. Soll der Endpunkt der Extremität dem Boden nähern, so wird der Winkel so verändert, dass die Ausrichtung des Kindelements sich der Senkrechten nähert. Wenn nicht, so wird der Winkel in die entgegengesetzte Richtung verändert. \todo{das passiert aber nicht}

% Tweaks
Je nach Ausgangsposition sieht das Ergebnis aber nicht unbedingt natürlich aus. Zum Beispiel kann es passieren, dass das Fußgelenk nicht gedreht wird, also der Fuß das Schienbein einfach verlängert und die Spitze des Fußes Bodenkontakt hat. Wenn dann die Oberseite des Fußes näher am Boden ist als die Unterseite, dann ist das keine sinnvolle Position.
Um so etwas zu verhindern, wird die Startposition der Extremität so gewählt, dass alle Gelenke stark angewinkelt sind. \todo{Abbildung}
Außerdem wird während der Iteration verboten, dass Knochen unterhalb der Bodenhöhe enden.
\todo{Zweiter Freiheitsgrad an Hüft- und Schultergelenk macht es schwieriger, da Winkel, je nach Winkel der anderen Richtung, vergrößert oder verkleinert werden muss um den Boden zu erreichen. Weggelassen. Bei mehr Anforderungen doch IK verwenden.}

% Änderung der Winkel und Wahrscheinlichkeiten
In jedem Schritt werden die Winkel um eine bestimmte Gradzahl verändert. Diese Gradzahl verkleinert sich mit jedem Schritt bis zu einer Minimalgröße. Zu Beginn werden die Winkel stark verändert um die grobe Ausrichtung des Beines festzulegen und in den kleiner werdenen Schritten wird die Extremität genauer ausgerichtet, so dass der Endpunkt zum Schluss auf dem Boden steht.
Zusätzlich wird nicht in jedem Schritt jeder Freiheitsgrad jedes Gelenks verändert. Für jeden Freiheitsgrad wird eine Wahrscheinlichkeit (kleiner als eins) festgelegt, dass dieser ausgewählt wird. Dadurch können bestimmte Richtungen oder Gelenke priorisiert werden um ein besseres Ergebnis zu erzielen. \todo{Was sind die guten Einstellungen? bzw braucht man dise Wkten überhaupt?}
\todo{Konkrete Einstellungen erwähnen (in Implementierungsdetails?)}

Evaluierung \todo{ausformulieren}
\begin{itemize}
 \item generierte Extremitäten mit echten Positionen vergleichen, diskutieren\\
    schlimmes Beispiel für Beinalgo: Brachiosaurus
 \item Restpose nicht klar definiert (vielleicht irgendwie in PCA einfügen? hier gab es keine guten Ideen; außerdem ist Position auf Skelettbildern teilweise auch eher beliebig)
 \item keine wissenschaftliche Argumentation für Änderungen am Beinalgo, nicht klar was Verbesserungen wären; Beine müssen für Animation, Muskelgeneration etc sowieso nochmal angefasst werden -> future work
\end{itemize}


%--------------------------------------
\section{Ansatzpunkte für Extremitäten}

Ansatzpunkte für Extremitäten sind natürlich zunächst der Hüftgürtel und der Schultergürtel. Um auch die Generierung fantastischer Tiere zu ermöglichen, ist es aber Möglich dies zu erweitern.

Eine einfache Möglichkeit ist hier zunächst die Anzahl der möglichen Extremitätenpaare von zwei auf vier zu erhöhen, indem einfach an der Hüfte und der Schulter jeweils zwei Paare ansetzen dürfen. Dafür wurden einfach an der Hüfte \bzw der Schulter mehrere Gelenke mit ein wenig Abstand angelegt, an denen Extremitäten ansetzen können.
Flügel und Arme dürfen hierbei weiterhin nur an der Schulter ansetzen, Beine und Flossen an beiden Stellen. Der Grund dafür ist, dass die meisten generierten Skelette seltsam wirken, wenn an der Hüfte Flügel oder Arme ansetzen und dafür an der Schulter Beine beginnen. Das liegt daran, dass existierende Tiere mit Flügeln oder Armen ihren Schwerpunkt im hinteren Bereich haben und sie auf den Hinterbeinen stehen.

% mehr Extremitätengürtel auf dem Rücken
Eine Überlegung war auch zwischen Schulter und Hüfte weitere Extremitätengürtel anzubringen. Das stellt sich aber als schwierig heraus. Die Wirbelsäule ist zwischen Hüfte und Schulter nach oben geschwungen und im Bauchraum befinden sich die meisten Organe des Tieres. Ein zusätzlicher Extremitätengürtel würde den Bauchraum einschränken. Außerdem wirkt dann auch die nach oben geschwungene Wirbelsäule anatomisch seltsam.
"`Verdoppelt"' man die Schwingung der Wirbelsäule und hängt einfach einen weiteren Rücken hinten oder vorne an, so wirkt es ebenso seltsam, da dann die "`Höcker"' der Wirbelsäule für das Tier wahrscheinlich nicht wirklich ein Vorteil sind und nur die Fortbewegung erschweren.

% zweiter Schultergürtel
Eine weitere Idee, die auch umgesetzt wurde, ist, eine Art Zentauren zu ermöglichen. Hat das Tier einen Hals, der lang genug ist, kann darauf ein weiterer Schultergürtel kurz unterhalb vom Kopf angebracht werden. An diesem Schultergürtel dürfen dann keine alle Arten von Extremitäten außer Beinen ansetzen. Das wirkt tatsächlich meist auch anatomisch einigermaßen sinnvoll.

%- - - - - - - - - - - - - - - - - - - 
\subsection{Anordnung der Extremitäten}

\begin{itemize}
 \item Anzahl der Extremitäten orientiert sich an Benutzereingabe
 \item falls keine Benutzereingabe, dann orientiert es sich an den von der PCA ausgespuckten Wahrscheinlichkeiten für Beine und Flügel
 \item die Position jeder Extremität wird zufällig aus der Menge der möglichen Positionen ausgewählt. Ist für eine Extremität kein Platz mehr, wird geschaut, ob andere Extremität an eine andere Position wechseln kann um Platz zu schaffen
 \item da Positionen nicht deterministisch kommt es bei Tieren mit mehreren möglichen Anordnungen zu unterschiedlichen Ergebnissen bei gleicher Eingabe
 \item da die Anzahl der Arme und Flossen nicht mit PCA erhoben kann sie für die Eingabebeispiele der PCA nicht rekonstruiert werden. Deshalb kann Benutzer hier eigene Angaben machen.
 \item nachdem Extremitäten zugeteilt wurden, werden sie nochmal "`verteilt"', damit nicht \zb zwei Beine an Hüfte aber keins an Schulter \todo{sinnvoll?}
 \item wenn es keine Angabe für die Anzahl der Flügel gibt, werden diese mit der PCA Wkt generiert, aber max ein Paar pro Schultergürtel; bei Beinen wir die Wkt (kann auch 2 sein) auf- oder abgerundet
 \item Arme werden ebenfalls mit Flügelwkt generiert (auch max 1 pro Schultergürtel)
 \item Flossen werden in komplett leere Extremitätengürtel generiert, wenn sie nicht zu lang wären
\end{itemize}



%--------------------------------
\section{Knochenmodelle einfügen}

Zunächst wird jeder terminale Knochen durch seine Bounding Box dargestellt.
\todo{es sind nicht wirklich Bounding Boxen, eher "`Proxyboxen"'}
Diese Boxen lassen sich aber leicht durch die 3D-Modelle der entsprechenden Knochen ersetzen. Dazu müssen die 3D-Modelle nur im .obj-Format vorliegen und folgenden Bedingungen entsprechen:

Das Modell ist korrekt an den Achsen ausgerichtet und so verzerrt, dass es einen Würfel mit 1(m) Kantenlänge in jeder Richtung möglichst gut ausfüllt.

Lässt man es hierbei bewenden, so ist es relativ schwierig herauszufinden wie man die einzelnen Knochen skalieren muss, dass sie an den Gelenken gut zusammenpassen. Außerdem ist es aufwändig herauszufinden wo die Gelenke an den Knochen ansetzen.

Setzt man sich dagegen etwas über den Gedanken der "`Bounding Box"' hinweg, so kann die Positionierung und Skalierung einfacher werden. Hier wurden, je nach Knochen, einige der folgenden Punkte umgesetzt. \todo{Beispielbilder}
\begin{itemize}
 \item Kleine Fortsätze, die nicht wirklich zur (optischen) Größe des Knochens beitragen, \zb die Fortsätze der Wirbel, ragen aus der Bounding Box heraus.
 
 \item Kantenlängen, die von "`außen"' vorgegeben werden, sind genau auf die Kantenlänge der Box skaliert (also 1). So beispielsweise die x-Länge der Wirbel, die auf der Wirbelsäule genau aneinender stoßen sollen. Dies können aber auch Längen sein, nur einen Teil des Knochens betreffen. Der Beinabstand an der Hüfte ist \zb kleiner als die komplette Breite der Hüfte. Es ist aber einfacher den Beinabstand anzugeben, als die Hüftbreite. Auch die Skalierung der Hüfte in x-Richtung ist zunächst nicht klar, aber wenn die Breite der zugehörigen Wirbel gegeben ist, ist auch klar, wie breit die Hüfte sein soll. Deshalb ist die Hüfte in x-Richtung so skaliert, dass der Teil, an dem der Wirbel ansetzt, schon die komplette Kantenlänge des Würfels ausfüllt. Bei Gelenken, die von der Breite her zusammenpassen sollen, ist dies auch sehr hilfreich. Aber das führt natürlich auch dazu, dass die "`Bounding Box"' nicht mehr viel mit der resultierenden Größe des Knochens zu tun haben muss.
 
 \item Kantenlängen, die nicht vorgegeben werden, sind einfacher passend zu bestimmen, wenn sie nicht komplett unabhängig von den anderen Raumrichtungen sind. Ist \zb die x- und y-Skalierung eines Knochens vorgegeben, und die Skalierung in z-Richtung soll nur möglichst gut dazu passen, so ist es sinnvoll, das 3D-Modell schon so zu speichern, dass die z-Richtung von einer der anderen Richtungen abhängt. Tut man dies nicht, so führt das relativ leicht dazu, dass die Knochen grundlos verzerrt werden.
\end{itemize}

Liegen die Modelle in diesem Format vor, können sie einfach eingelesen werden und anhand der Skalierung der Bounding Box skaliert werden.
Hier wurden vor allem Modelle von menschlichen Knochen verwendet, da sie leichter verfügbar sind. Manche Knochen sind jedoch auch von anderen Tieren. Das führt \zb bei dem verwendeten Unterarmknochen des Pferdes dazu, dass er etwas überdimensionierte Fortsätze am Ellenbogen bekommen, wenn man ihn großskaliert. Das liegt daran, dass dieser Knochen beim Pferd eigentlich relativ kurz ist.

% Ausrichtung der Knochen
Eine Schwierigkeit daran Modelle in der oben genannten Form herzustellen ist, dass nicht unbedingt sofort klar ist, wie die Knochen ausgerichtet werden müssen. Die Hüfte muss \zb so ausgerichtet werden, dass der Anfangs- und Endpunkt der durchgehenden Wirbelsäule auf gleicher Höhe liegen, damit nachfolgende Wirbel auch richtig anschließen. (Das funktioniert natürlich nur, weil die Wirbelsäule an der Stelle der Hüfte quasi gerade ist.) \todo{wg Lizenz Pferdehüfte verwendet, die nicht mit Wirbelsäule verwachsen ist}
Auch die Positionierung der Rippen und des Oberarms in Kombination mit dem Unterarm ist anspruchsvoll. Dabei hilft es 3D-Modelle zu haben, in denen die anderen Knochen auch schon vorhanden sind, um sich die Ausrichtung abzuschauen. Außerdem können Bilder von Skeletten zu Rate gezogen werden. Und zuletzt muss man die genaue Positionierung einfach testen.

% Gelenke korrekt ausrichten
Zusätzlich muss beachtet werden wie die Knochen aneinander anschließen \bzw wie sie für die entsprechenden Gelenke korrekt positioniert sind. Das erfordert etwas "`finetuning"'. Für jeden Knochen sind dafür in Abhängigkeit zur Bounding Box zwei Offsets gespeichert: das Offset zu dem Gelenk, das ihn mit seinem Elternknochen verbindet und das Offset zu dem Gelenk, das ihn mit seinem Kindknochen verbindet (oder mehrere, falls vorhanden). Dies sorgt dafür, dass die Positionierung der Knochen stimmt, egal wie groß sie sind. Ist ein Knochen sehr groß und ein anschließender sehr klein (oder anders herum), so kommt es natürlich trotzdem vor, dass die Gelenke nicht wirklich ineinander passen. Für solche Situationen bräuchte man verschiedene 3D-Modelle, die je nach Gegebenheit eingesetzt werden.

% Köpfe
Da der Kopf \bzw der Schädelknochen im Gegensatz zu anderen Knochen bei Wirbeltieren sehr stark variiert, ist es sinnvoll mehrere Schädelknochen zur Auswahl zu haben. Geht die Menge an verfügbaren Schädelknochen über "`Tier mit Flügeln"' und "`Tier ohne Flügel"' hinaus, so ist es außerdem sinnvoll die Auswahl des passenden Schädelknochens dem Nutzer zu überlassen.

% Hände und Füße
Bei Händen und Füßen ist das Problem ebenso, dass es sehr viele verschiedene Ausprägungen davon gibt. Hier lassen sie sich jedoch grob nach Extremitätentyp unterscheiden. Diese Unterscheidung kann aber beliebig fein sein. Hier wurde nur nach Flügel, Flosse \todo{?} und Extremität mit Bodenkontakt unterschieden. Und bei Extremitäten mit Bodenkontakt wurde nochmals danach unterschieden wie flach der Fuß oder die Hand auf dem Boden aufkommt (bei weniger als $45^\circ$ ist es eine menschliche Hand, sonst ein Huf). \todo{ggf. an Implementierung anpassen}
\todo{erwähnen, dass Knochen für Flügel"`hand"' nicht vollständig}

\todo{Schwanzspitze: ein Modell für drei Wirbel. Das sieht bei wenigen Wirbeln komisch abgeknickt aus}

% mehrere Extremitäten an einer Schulter/Hüfte
Setzten mehrere Extremitäten an einer Schulter oder einer Hüfte an, so ist ein "`normales"' Modell der Schulter/Hüfte nicht mehr ausreichend, da nicht genug Gelenke vorhanden sind. Dieses Problem wurde an der Schulter so behoben, dass einfach mehrere Schulterblätte mit einem gewissen Abstand generiert wurden. Für die Hüfte wurde ein kombiniertes 3D-Modell aus zwei einzelnen Hüften erstellt, an dem nun zwei Gelenke vorhanden sind.

\todo{3D-Modelle leider nicht ganz einfach Austauschbar wegen Offsets -> erkläre wie Offsets eingelesen werden müssten}
\todo{Wenn Muskeln generiert werden sollen, müssen noch mehr Ansatzpunkte bestimmt werden und Positionierung der Knochen hängt von Muskeln ab -> future work}

%------------------------
\section{Ergebnisskelette}

\begin{itemize}
 \item Einheiten der PCA für Koordinaten $[0, 1000]$, deshalb sind die Wirbelsäulen der generierten Skelette auch in diesem Rahmen. Blender interpretiert eine Einheit als $1$m. Deshalb wirken sie sehr groß.
 
 \item Die Abmessungen der Knochen in die verschiedenen Richtungen ist bei den meisten Knochen relativ beliebig gewählt und oft auch immer gleich (außer bei Längen, die von PCA vorgegeben sind). Dafür gibt es keine biologische oder anatomische Grundlage. Man könnte hier sicherlich noch mehr machen (mehr Zufall, mehr anatomisch korrekt etc.)
 \todo{in Implementierungsdetails aufzählen was an Abmessungen alles beliebig (oder auch weniger beliebig) festgelegt wurde.}
\end{itemize}

%------------------------------------------
\section{Speichern und Laden von Skeletten}

Um ein Skelett zu speichern reicht es jene Metadaten herauszuschreiben, aus denen der Algorithmus das Skelett wieder reproduzieren kann. Zu den benötigten Daten gehören die Daten, die die PCA liefert (Position der Wirbelsäule, Länge der Knochen in den Extremitäten, Gewicht) und Daten, die durch den Algorithmus generiert werden (Anzahl der Wirbel, Intervall auf der Wirbelsäule, in dem sich Rippen befinden, Anzahl und Art der Extremitäten an den jeweiligen Extremitätengürteln und die Winkel an Gelenken der Extremitäten, Art des einzusetzenden Kopfes).
Das alles wird in ein paar wenigen Java-Klassen gebündelt und über Java-Serialisierung in eine Textdatei geschrieben. Diese Datei kann dann wieder eingelesen werden um die Klassen wieder herzustellen.

Das funktioniert natürlich nur über das implementierte Programm und liefert keine Metainformationen zu dem generierten Skelett nach außen.
\todo{Ideen für Zusatzinfos, die sinnvoll sein könnten, aufschreiben / future work (zB Knochenhierarchie); wären aber wsh nicht super hilfreich, weil für Animation etc. sowieso nochmal alles angefasst werden muss}

\todo{Möglichkeiten Zusatzinfos abzuspeichern (fbx, alembic, universal scene description \url{https://graphics.pixar.com/usd/docs/index.html}), python script für Blender etc. um Knochen anzulegen}

Erzeugung von Variationen \todo{ausformulieren}
\begin{itemize}
 \item was genau wird variiert
 \item PCA Daten normalverteilt variieren (Verteilung bleibt Gauß, Erwartungswert bleibt gleich, Überlegungen dazu wie sie sich ändert durch Aufaddieren der Variation / Faltung mit anderem Gauß)
 \item Variation der Daten, die nicht von PCA abhängen, beschreiben; generische Algorithmen zitieren (auch hinzufügen und löschen oder ändern von Features um Variationen/Verbesserungen zu erzeugen)
\end{itemize}

