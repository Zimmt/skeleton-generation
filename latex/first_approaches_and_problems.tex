%-----------------------------------
%-----------------------------------
\chapter{Erste Ansätze und Probleme}

%------------------------------------
\section{Bestandteile eines Skeletts}

Grundelemente aus denen ein Skelett besteht: Knochen, Gelenke + jeweils (relative) Positionen und Orientierungen (Knochen zunächst Rechtecke (siehe Knochenmodelle))
\todo{erwähnen, dass keine aufrechteren Tiere als T-Rex oder Känguru generiert werden (keine Menschen)}

\begin{itemize}
 \item Ein Skelett besteht aus Knochen, Gelenken und deren (relativen) Positionen und Orientierungen. (Verwendung von homogenen Transformationsmatrizen)
 
 \item Ein Knochen ist im einfachsten Fall ein Zylinder (Strecke) mit Länge und Radius, kann aber auch eine konvexe Kurve mit einem Radius sein.
 
 \item Ein Gelenk hat keine Ausdehnung (?). Es ist das Verbindungsstück zwischen zwei oder mehr Knochen und legt fest wie die Knochen sich relativ zueinander bewegen können. Werden mehr als zwei Knochen verbunden ist es einfach eine feste Verbindung.
 \todo{Schaubilder für verschiedene Gelenkarten?}
 
 \item Terminale zunächst als Bounding Box. Später werden sie durch Knochenmodelle ersetzt. Für jeden möglichen Knochen wird ein Modell gespeichert. Diese Modelle werden so skaliert, dass sie einen 1x1-Würfel ausfüllen. So kann dann später leicht die Skalierung auf diesen Würfel übertragen werden.
 
 \item Knochen: Bounding Box bzw. Skalierung in drei Raumrichtungen, Position, Orientierungen und Name bzw. ID (wird für Ersetzung benötigt).
 
 \item Knochen in Hierarchie anordnen und Position und Rotation relativ zu Elternelement angeben, da damit Erzeugung von Kindelementen einfacher $\rightarrow$ verwende homogene Koordinaten.\\
 Bei der Erzeugung von Elementen auf der Wirbelsäule ist das nicht einfacher. Aber alle Algorithmen, die Tiere animieren brauchen eine Hierarchie, deshalb ist das wichtig. \todo{Quelle, Beispiele}
 
 \item Die Knochen sollen eine Hierarchie (Baum) bilden. Am besten ist ein Knochen in der Nähe es Schwerpunkts (oft wird die Hüfte verwendet). Die Hüfte gibt es aber nicht immer. Verwende ersten Wirbel nach dem Brustkorb (fals existent).
 
 \item Gelenke: Position, Bewegungseinschränkungen für alle drei Richtungen. In einer Hierarchie verbindet das Gelenk ein Kindelement mit seinem Elternelement. Gelenk ist hier ein Punkt, der im Koordinatensystem des Elternelements angegeben wird, und um den das Koordinatensystem des Kindelements gedreht werden kann/soll.\\
 Das macht es sinnvoll ein Kindelement erst zu erzeugen, wenn das dazugehörige Elternelement terminal ist. Sonst weiß man noch nicht so genau wie das Gelenk aussehen soll. Mit etwas mehr Aufwand lässt sich das aber auch im Nachhinein noch anpassen\dots\\
 \todo{Bewegungseinschränkungen für bestimmtes Gelenk bei jedem Tier gleich? Wie Einschränkungen erzeugen?}, %TODO 
 
 \item Erzeugung von Kindelementen nicht nur von direktem Elternteil abhängig $\rightarrow$ Abhängigkeiten zwischen entfernten Teilen nötig; auch für Balance nützlich. "`Kommunikation"' über Überobjekt, das auch alle Einzelteile speichert. (SkeletonGenerator)
\end{itemize}


%-----------------------------
\section{Aufbau als Grammatik}

\begin{itemize}
  \item Aufbau als Grammatik (am Ende aber für jedes Nichtterminal genau eine Regel)
  \item Aufbau der (Nicht)terminale als Baum, nur Terminale können Kinder haben, Reihenfolge der Anwendung der Regeln sollte beliebig sein
  \item Wachstum unter Berücksichtigung verschiedener Randbedingungen wie Bodenposition, Anzahl Extremitäten etc.\ Bounding Box für Nichtterminale ist dafür aber nicht nötig.
  \item Symmetrie der Skelette (spiegle Elemente zum Schluss)
  \item Repräsentation des Zustands als Hierarchie von einzelnen Komponenten (terminale sowie nichtterminale)
 \end{itemize}
 
 \begin{itemize}
  \item Iterative Erzeugung eines Skeletts durch eine probabilistische kontextfreie (?) Grammatik, die so erweitert ist, dass sie nicht ein einfaches Wort erzeugt, sondern einen Baum von Zeichen (nötig für Extremitäten). Verwendung von paramterischen L-Systemen \cite{Paramteric_L-Systems} könnte sinnvoll sein.
  
  \item Regeln sind nicht wirklich eine Grammatik, da fast jedes nichtterminale Literal nur einmal vorkommt, wenn es für jedes Körperteil andere Regeln gibt. Oder ist es möglich so zu abstrahieren, dass z.B. Arme und Beine den gleichen/ähnlichen Regeln unterliegen? Ist das sinnvoll? \todo{Abstraktionsgrad, Art der Regeln}\\
  \todo{Graphen zur Visualisierung der Regeln einfügen}
  Außerdem ist das Skelett nicht unbedingt zusammenhängend (siehe Biologie). $\rightarrow$ Darauf achten, dass das nicht von Algorithmus verlangt wird
  
  \item Brustbein sorgt dafür, dass Skelett nicht mehr baumartig $\rightarrow$ erstmal weglassen, ist wahrscheinlich auch nicht unglaublich relevant
 \end{itemize}

 

%---------------------
\section{Pose des Skeletts}

Ein Skelett ist nicht nur eine Hierarchie von Knochen, sondern wirkt auch wesentlich durch seine Pose. Das Ziel sollte also sein das Skelett in einer natürlich wirkenden Pose darzustellen, eine Pose, die das dargestellte Tier auch einnehmen würde.

% erster Ansatz: Drehmomente
Ein erster Ansatz könnte sein in jedem Schritt auszurechnen, ob das Skelett ausbalanciert ist. Dazu benötigt man den Schwerpunkt des Körpers und Position und Gewicht der Knochen.
Damit kann man dann die Drehmomente der einzelnen Knochen um den Schwerpunkt berechnen. Addieren sie sich alle zu null auf, ist das Skelett im Gleichgewicht.

% Probleme
Hierbei tauchen gleich mehrere Fragen auf:
\begin{description}
 \item[Gewicht] Wie wird das Gewicht der Knochen bestimmt? Es wird das Gewicht aller Elemente, oder zumindest das aller terminalen Elemente, benötigt. Wie wird es festgelegt? Hängt es von der Größe der Knochen ab? Außerdem müsste das ganze Gewicht des Tieres, nicht nur das der Knochen berücksichtigt werden. Wie kann bestimmt werden wieviel anderes Gewebe an einem Knochen hängt?
 
 \item[Schwerpunkt] Wo liegt der Schwerpunkt? Wird am Anfang eine große Bounding Box für das Skelett festgelegt, in deren Mitte der Schwerpunkt liegt? Verändert der Schwerpunkt seine Position je nach dem was generiert wird?
 
 \item[Gleichgewicht] Wann wird überprüft ob sich das Skelett im Gleichgewicht befindet? Ist das Gleichgewicht eine Invariante, die während der Generierung aufrecht erhalten werden soll? Oder wird es erst am Ende geprüft? Was passiert, wenn sich das Skelett nicht im Gleichgewicht befindet?
\end{description}
 
% Warum wird anderer Ansatz verwendet
Das Hauptproblem ist, dass nicht klar ist wie das Gewicht der einzelnen Körperteile bestimmt werden soll ohne zusätzlich zu den Knochen auch noch anderes Gewebe wie Muskeln oder Eingeweide zu betrachten. Deshalb wurde ein anderer Ansatz erarbeitet.
 
% Wirbelsäule als Startpunkt
Die Wirbelsäule ist ein zentraler Teil des Wirbeltierkörpers und bestimmt wesentlich das Aussehen des Skeletts.
Ihre Form variiert von relativ gerade, \zb bei Fischen oder Schlangen, bis hin zu starkt geschwungenen Hälsen, \va bei Vögeln, und langen Schwänzen, \zb bei Mäusen. Außerdem zeigt sie wie aufrecht ein Tier sich hält. Große Unterschiede sind hier zwischen Fischen und auch Vierbeinern gegenüber Vögeln zu beobachten, da Vögel nur auf ihren Hinterbeinen stehen und deshalb ihr Schwerpunkt nach hinten verschoben ist. Es gibt aber auch aufrechtere Exemplare unter den "`Vierbeinern"' wie beispielsweise das Känguru oder der Tyrannosaurus Rex.\todo{Beispielbilder von Wirbelsäulen} \\
Der Mensch ist natürlich auch ein Beispiel für ein sehr aufrechtes Wirbeltier. Seine Haltung unterscheidet sich aber so stark von der der anderen Wirbeltiere, dass er zunächst außen vor gelassen werden soll.

Viele Knochen setzen direkt an der Wirbelsäule an, wie \zb der Kopf, die Rippen oder die Hüfte. Durch die Wirbelsäule wird also schon sehr viel vorgegeben.
Deshalb eignet sie sich sehr gut als Startpunkt für die Generierung eines Skeletts. Ausgehend von ihr kann dann der Rest des Skeletts "`wachsen"'.

% Wie Lage der Wirbelsäule bestimmen?
Wie soll aber nun die Lage der Wirbelsäule bestimmt werden?\\
Hierfür schien es sinnvoll viele Beispiele zu betrachten und Zusammenhänge zwischen verschiedenen Eigenschaften der Tiere und dem Verlauf der Wirbelsäule zu suchen.
Ein geeignetes Werkzeug hierfür ist die \emph{Principal Component Analysis}, die im folgenden Kapitel beschrieben wird.

