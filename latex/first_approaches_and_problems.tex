%-----------------------------------
%-----------------------------------
\chapter{Erste Ansätze und Probleme}

%------------------------------------
\section{Bestandteile eines Skeletts}

Grundelemente aus denen ein Skelett besteht: Knochen, Gelenke + jeweils (relative) Positionen und Orientierungen (Knochen zunächst Rechtecke (siehe Knochenmodelle))

\begin{itemize}
 \item Ein Skelett besteht aus Knochen, Gelenken und deren (relativen) Positionen und Orientierungen.
 
 \item Ein Knochen ist im einfachsten Fall ein Zylinder (Strecke) mit Länge und Radius, kann aber auch eine konvexe Kurve mit einem Radius sein.
 
 \item Ein Gelenk hat keine Ausdehnung (?). Es ist das Verbindungsstück zwischen zwei oder mehr Knochen und legt fest wie die Knochen sich relativ zueinander bewegen können. Werden mehr als zwei Knochen verbunden ist es einfach eine feste Verbindung.
 
 \item Terminale zunächst als Bounding Box. Später werden sie durch Knochenmodelle ersetzt. Für jeden möglichen Knochen wird ein Modell gespeichert. Diese Modelle werden so skaliert, dass sie einen 1x1-Würfel ausfüllen. So kann dann später leicht die Skalierung auf diesen Würfel übertragen werden.
 
 \item Knochen: Bounding Box bzw. Skalierung in drei Raumrichtungen, Position, Orientierungen und Name bzw. ID (wird für Ersetzung benötigt).
 
 \item Knochen in Hierarchie anordnen und Position und Rotation relativ zu Elternelement angeben, da damit Erzeugung von Kindelementen einfacher $\rightarrow$ verwende homogene Koordinaten.\\
 Bei der Erzeugung von Elementen auf der Wirbelsäule ist das nicht einfacher. Aber alle Algorithmen, die Tiere animieren brauchen eine Hierarchie, deshalb ist das wichtig. \todo{Quelle, Beispiele}
 
 \item Gelenke: Position, Bewegungseinschränkungen für alle drei Richtungen. In einer Hierarchie verbindet das Gelenk ein Kindelement mit seinem Elternelement. Gelenk ist hier ein Punkt, der im Koordinatensystem des Elternelements angegeben wird, und um den das Koordinatensystem des Kindelements gedreht werden kann/soll.\\
 Das macht es sinnvoll ein Kindelement erst zu erzeugen, wenn das dazugehörige Elternelement terminal ist. Sonst weiß man noch nicht so genau wie das Gelenk aussehen soll. Mit etwas mehr Aufwand lässt sich das aber auch im Nachhinein noch anpassen\dots\\
 \todo{Bewegungseinschränkungen für bestimmtes Gelenk bei jedem Tier gleich? Wie Einschränkungen erzeugen?}, %TODO 
 
 \item Erzeugung von Kindelementen nicht nur von direktem Elternteil abhängig $\rightarrow$ Abhängigkeiten zwischen entfernten Teilen nötig; auch für Balance nützlich. "`Kommunikation"' über Überobjekt, das auch alle Einzelteile speichert. (SkeletonGenerator)
\end{itemize}


%-----------------------------
\section{Aufbau als Grammatik}

\begin{itemize}
  \item Aufbau als Grammatik (am Ende aber für jedes Nichtterminal genau eine Regel)
  \item Aufbau der (Nicht)terminale als Baum, nur Terminale können Kinder haben, Reihenfolge der Anwendung der Regeln sollte beliebig sein
  \item Symmetrie der Skelette (spiegle Elemente zum Schluss)
  \item Repräsentation des Zustands als Hierarchie von einzelnen Komponenten (terminale sowie nichtterminale)
 \end{itemize}
 
 \begin{itemize}
  \item Iterative Erzeugung eines Skeletts durch eine probabilistische kontextfreie (?) Grammatik, die so erweitert ist, dass sie nicht ein einfaches Wort erzeugt, sondern einen Baum von Zeichen (nötig für Extremitäten). Verwendung von paramterischen L-Systemen \cite{Paramteric_L-Systems} könnte sinnvoll sein.
  
  \item Regeln sind nicht wirklich eine Grammatik, da fast jedes nichtterminale Literal nur einmal vorkommt, wenn es für jedes Körperteil andere Regeln gibt. Oder ist es möglich so zu abstrahieren, dass z.B. Arme und Beine den gleichen/ähnlichen Regeln unterliegen? Ist das sinnvoll? \todo{Abstraktionsgrad, Art der Regeln}\\
  \todo{Graphen zur Visualisierung der Regeln einfügen}
  Außerdem ist das Skelett nicht unbedingt zusammenhängend (siehe Biologie). $\rightarrow$ Darauf achten, dass das nicht von Algorithmus verlangt wird
  
  \item Brustbein sorgt dafür, dass Skelett nicht mehr baumartig $\rightarrow$ erstmal weglassen, ist wahrscheinlich auch nicht unglaublich relevant
 \end{itemize}

 

%---------------------
\section{"`Restpose"'}

\begin{itemize}
  \item Tier sollte funktional aussehen (was für Regeln sind da nötig?), Wie ausbalancierte Skelette generieren? Wie Gewicht der Teile bestimmen?
  \item Form der Wirbelsäule bestimmt wesentlich das Aussehen $\rightarrow$ Startpunkt
 \end{itemize}
 
\begin{itemize}
  \item Regeln wichtig, die dafür sorgen, dass das Tier am Ende auch funktional ist.
  \item Darstellung des Skeletts in "`sinnvoller"' Pose.
  \item Beachte, dass Tier nicht umfällt: berechne Drehmomente um Kontaktpunkte mit Boden. Dafür sind aber Muskeln bzw.\ Masse wichtig $\rightarrow$ Schätzung der Masse anhand der Knochengröße bzw.\ Größe der Bounding Box?
  
  \item Lage der Wirbelsäule am Anfang festlegen, da sie der zentrale Teil des Körpers ist und die Lage der anderen Körperteile grob festlegt. Form variiert von Gerade (Fisch) bis geschwungener Hals und Schwanz (z.B. Giraffe). Mensch passt hier nicht ganz ins Schema $\rightarrow$ weglassen?
  
  \item Position der Wirbelsäule wird zu Beginn durch ein, oder mehrere, Beziérkurve(n) festgelegt. Außerdem wird gestgelegt wo Elemente ansetzen, die nicht einfach Wirbel sind, \zb Extremitäten. Von diesen Punkten aus, kann das Skelett dann weiter "`wachsen"'. Diese Positionierung der Wirbelsäule legt die Länge und die Höhe des Tieres fest. Außerdem natürlich den Verlauf der Wirbelsäule, also \zb auch die Länge des Halses.\\
  Regel, die zunächst sinnvoll klingt: Ansatzpunkte der Extremitäten an der Wirbelsäule können nur in Bereichen der Kurve sein, an denen die Steigung nahe null ist.

  \item Wachstum unter Berücksichtigung verschiedener Randbedingungen wie Bodenposition, Anzahl Extremitäten etc.\ Bounding Box für Nichtterminale ist dafür aber nicht nötig.
  
  \item Die Knochen sollen eine Hierarchie (Baum) bilden. Am besten ist ein Knochen in der Nähe es Schwerpunkts (oft wird die Hüfte verwendet). Die Hüfte gibt es aber nicht immer. Verwende ersten Wirbel nach dem Brustkorb (fals existent).
\end{itemize}

%- - - - - - - - - -
\subsection{Balance} 

Berechnung der Blance: Drehmomente der Knochen (benötigt Position der Knochen und Gewicht) $\rightarrow$ zunächst nicht nötig, da Position  der Wirbelsäule zu beginn mit Beziérkurve festgelegt wird

  \begin{itemize}
   \item Drehmoment $\vec{M} = \vec{r} \times \vec{F}$, mit Ortsvektor $\vec{r}$ vom Bezugspunkt des Drehmoments zum Angriffspunkt der Kraft $\vec{F}$. Betrag des Drehmoments $M = r \cdot F \cdot \textrm{sin}(\alpha)$, mit dem Winkel $\alpha$ zwischen $\vec{F}$ und $\vec{r}$.\\
   (\url{https://de.wikipedia.org/wiki/Drehmoment})
   
   \item Kraft, die auf Knochen wirkt: resultierende Kraft der Streckenlast, die durch zugewiesenes Gewicht erzeugt wird. Der Einfachheit halber einfach rechteckige Streckenlast annehmen, dann greift Kraft am Mittelpunkt der Strecke und $\vec{F} = \frac{1}{2} m g$\footnote{Die Formel zu rechteckigen Streckenlasten, die man so findet ist $\vec{F} = \frac{1}{2} L q_0$, mit Länge der Strecke $L$ und den auf der Strecke verteilten Einzellasten $q_0$, aber $L q_0 = m g$. Daraus folgt obige Formel.}, mit Masse $m$ und Erdbeschleunigung $g$\\ (\url{https://www.der-wirtschaftsingenieur.de/index.php/streckenlast/}, \url{https://www.youtube.com/watch?v=T1Sf8GPD4dc})
   
   \item Reduzierung des Problems auf zwei Dimensionen, da Körper so aufgebaut ist, dass er nur nach vorne oder hinten umkippen kann und nicht zur Seite.
   
   \item Damit Körper immer in Balance ist, muss auch nicht terminalen Teilen ein Drehmoment (und damit auch implizit ein Gewicht) zugewiesen werden. Werden aus diesen Nichtterminalen dann neue Teile generiert, so muss das Drehmoment bzw.\ die Summe der einzelnen Drehmomente gleich bleiben.
   
   \item Der Bezugspunkt für die Berechnung der Drehmomente ist der Schwerpunkt des Körpers, da sich dort alle durch das Eigengewicht wirkenden Drehmomente des Körpers zu null aufaddieren.\\ (\url{https://www.grund-wissen.de/physik/mechanik/drehmoment-und-gleichgewicht.html})\\
   Das heißt zu Beginn der Generierung muss der Schwerpunkt des Körpers festgelegt werden, damit die Drehmomente berechnet werden können. Das ist am Anfang aber sehr einfach, da das initiale Element die Bounding Box des Gesamten Skeletts ist. Also ist der Schwerpunkt der Mittelpunkt.
   
  \end{itemize}
