%-----------------------
%-----------------------
\chapter{Knochenmodelle}

%--------------------------------
\section{Knochenmodelle einfügen}

Zunächst wird jeder terminale Knochen durch seine Bounding Box dargestellt.
\todo{es sind nicht wirklich Bounding Boxen, eher "`Proxyboxen"'}
Diese Boxen lassen sich aber leicht durch die 3D-Modelle der entsprechenden Knochen ersetzen. Dazu müssen die 3D-Modelle nur im .obj-Format vorliegen und folgenden Bedingungen entsprechen:

Das Modell ist korrekt an den Achsen ausgerichtet und so verzerrt, dass es einen Würfel mit 1(m) Kantenlänge in jeder Richtung möglichst gut ausfüllt.

Lässt man es hierbei bewenden, so ist es relativ schwierig herauszufinden wie man die einzelnen Knochen skalieren muss, dass sie an den Gelenken gut zusammenpassen. Außerdem ist es aufwändig herauszufinden wo die Gelenke an den Knochen ansetzen.

Setzt man sich dagegen etwas über den Gedanken der "`Bounding Box"' hinweg, so kann die Positionierung und Skalierung einfacher werden. Hier wurden, je nach Knochen, einige der folgenden Punkte umgesetzt. \todo{Beispielbilder}
\begin{itemize}
 \item Kleine Fortsätze, die nicht wirklich zur (optischen) Größe des Knochens beitragen, \zb die Fortsätze der Wirbel, ragen aus der Bounding Box heraus.
 
 \item Kantenlängen, die von "`außen"' vorgegeben werden, sind genau auf die Kantenlänge der Box skaliert (also 1). So beispielsweise die x-Länge der Wirbel, die auf der Wirbelsäule genau aneinender stoßen sollen. Dies können aber auch Längen sein, nur einen Teil des Knochens betreffen. Der Beinabstand an der Hüfte ist \zb kleiner als die komplette Breite der Hüfte. Es ist aber einfacher den Beinabstand anzugeben, als die Hüftbreite. Auch die Skalierung der Hüfte in x-Richtung ist zunächst nicht klar, aber wenn die Breite der zugehörigen Wirbel gegeben ist, ist auch klar, wie breit die Hüfte sein soll. Deshalb ist die Hüfte in x-Richtung so skaliert, dass der Teil, an dem der Wirbel ansetzt, schon die komplette Kantenlänge des Würfels ausfüllt. Bei Gelenken, die von der Breite her zusammenpassen sollen, ist dies auch sehr hilfreich. Aber das führt natürlich auch dazu, dass die "`Bounding Box"' nicht mehr viel mit der resultierenden Größe des Knochens zu tun haben muss.
 
 \item Kantenlängen, die nicht vorgegeben werden, sind einfacher passend zu bestimmen, wenn sie nicht komplett unabhängig von den anderen Raumrichtungen sind. Ist \zb die x- und y-Skalierung eines Knochens vorgegeben, und die Skalierung in z-Richtung soll nur möglichst gut dazu passen, so ist es sinnvoll, das 3D-Modell schon so zu speichern, dass die z-Richtung von einer der anderen Richtungen abhängt. Tut man dies nicht, so führt das relativ leicht dazu, dass die Knochen grundlos verzerrt werden.
\end{itemize}

Liegen die Modelle in diesem Format vor, können sie einfach eingelesen werden und anhand der Skalierung der Bounding Box skaliert werden.
Hier wurden vor allem Modelle von menschlichen Knochen verwendet, da sie leichter verfügbar sind. Manche Knochen sind jedoch auch von anderen Tieren. Das führt \zb bei dem verwendeten Unterarmknochen des Pferdes dazu, dass er etwas überdimensionierte Fortsätze am Ellenbogen bekommen, wenn man ihn großskaliert. Das liegt daran, dass dieser Knochen beim Pferd eigentlich relativ kurz ist.

% Ausrichtung der Knochen
Eine Schwierigkeit daran Modelle in der oben genannten Form herzustellen ist, dass nicht unbedingt sofort klar ist, wie die Knochen ausgerichtet werden müssen. Die Hüfte muss \zb so ausgerichtet werden, dass der Anfangs- und Endpunkt der durchgehenden Wirbelsäule auf gleicher Höhe liegen, damit nachfolgende Wirbel auch richtig anschließen. (Das funktioniert natürlich nur, weil die Wirbelsäule an der Stelle der Hüfte quasi gerade ist.) \todo{wg Lizenz Pferdehüfte verwendet, die nicht mit Wirbelsäule verwachsen ist}
Auch die Positionierung der Rippen und des Oberarms in Kombination mit dem Unterarm ist anspruchsvoll. Dabei hilft es 3D-Modelle zu haben, in denen die anderen Knochen auch schon vorhanden sind, um sich die Ausrichtung abzuschauen. Außerdem können Bilder von Skeletten zu Rate gezogen werden. Und zuletzt muss man die genaue Positionierung einfach testen.

% Gelenke korrekt ausrichten
Zusätzlich muss beachtet werden wie die Knochen aneinander anschließen \bzw wie sie für die entsprechenden Gelenke korrekt positioniert sind. Das erfordert etwas "`finetuning"'. Für jeden Knochen sind dafür in Abhängigkeit zur Bounding Box zwei Offsets gespeichert: das Offset zu dem Gelenk, das ihn mit seinem Elternknochen verbindet und das Offset zu dem Gelenk, das ihn mit seinem Kindknochen verbindet (oder mehrere, falls vorhanden). Dies sorgt dafür, dass die Positionierung der Knochen stimmt, egal wie groß sie sind. Ist ein Knochen sehr groß und ein anschließender sehr klein (oder anders herum), so kommt es natürlich trotzdem vor, dass die Gelenke nicht wirklich ineinander passen. Für solche Situationen bräuchte man verschiedene 3D-Modelle, die je nach Gegebenheit eingesetzt werden.

% Köpfe
% Köpfe sind kompliziert $\rightarrow$ Auswahl an Köpfen bereitstellen (evtl. leicht skalier-/verformbar oder ineinander überführbar)
Da der Kopf \bzw der Schädelknochen im Gegensatz zu anderen Knochen bei Wirbeltieren sehr stark variiert, ist es sinnvoll mehrere Schädelknochen zur Auswahl zu haben. Geht die Menge an verfügbaren Schädelknochen über "`Tier mit Flügeln"' und "`Tier ohne Flügel"' hinaus, so ist es außerdem sinnvoll die Auswahl des passenden Schädelknochens dem Nutzer zu überlassen.



% Hände und Füße
Bei Händen und Füßen ist das Problem ebenso, dass es sehr viele verschiedene Ausprägungen davon gibt. Hier lassen sie sich jedoch grob nach Extremitätentyp unterscheiden. Diese Unterscheidung kann aber beliebig fein sein. Hier wurde nur nach Flügel, Flosse \todo{?} und Extremität mit Bodenkontakt unterschieden. Und bei Extremitäten mit Bodenkontakt wurde nochmals danach unterschieden wie flach der Fuß oder die Hand auf dem Boden aufkommt (bei weniger als $45^\circ$ ist es eine menschliche Hand, sonst ein Huf). \todo{ggf. an Implementierung anpassen}
\todo{erwähnen, dass Knochen für Flügel"`hand"' nicht vollständig}

\todo{Schwanzspitze: ein Modell für drei Wirbel. Das sieht bei wenigen Wirbeln komisch abgeknickt aus}

% mehrere Extremitäten an einer Schulter/Hüfte
Setzten mehrere Extremitäten an einer Schulter oder einer Hüfte an, so ist ein "`normales"' Modell der Schulter/Hüfte nicht mehr ausreichend, da nicht genug Gelenke vorhanden sind. Dieses Problem wurde an der Schulter so behoben, dass einfach mehrere Schulterblätte mit einem gewissen Abstand generiert wurden. Für die Hüfte wurde ein kombiniertes 3D-Modell aus zwei einzelnen Hüften erstellt, an dem nun zwei Gelenke vorhanden sind.

\todo{3D-Modelle leider nicht ganz einfach Austauschbar wegen Offsets -> erkläre wie Offsets eingelesen werden müssten}
\todo{Wenn Muskeln generiert werden sollen, müssen noch mehr Ansatzpunkte bestimmt werden und Positionierung der Knochen hängt von Muskeln ab -> future work}


%------------------------
\section{Ergebnisskelette}

\begin{itemize}
 \item Einheiten der PCA für Koordinaten $[0, 1000]$, deshalb sind die Wirbelsäulen der generierten Skelette auch in diesem Rahmen. Blender interpretiert eine Einheit als $1$m. Deshalb wirken sie sehr groß.
 
 \item Die Abmessungen der Knochen in die verschiedenen Richtungen ist bei den meisten Knochen relativ beliebig gewählt und oft auch immer gleich (außer bei Längen, die von PCA vorgegeben sind). Dafür gibt es keine biologische oder anatomische Grundlage. Man könnte hier sicherlich noch mehr machen (mehr Zufall, mehr anatomisch korrekt etc.)
 \todo{in Implementierungsdetails aufzählen was an Abmessungen alles beliebig (oder auch weniger beliebig) festgelegt wurde.}
\end{itemize}

