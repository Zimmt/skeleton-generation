\chapter{Bisherige Arbeiten}

%-------------
\section{Ziva}

\begin{itemize}
 \item Ziva VFX Maya Plugin zur Erstellung von Charakteren und Simulation von biomechanischen Bewegungen \url{https://zivadynamics.com/}
 \item Charaktererstellung in Ziva beginnt mit der Modellierung des Skeletts. Knochen mit Animationen werden als Alembic-Datei gespeichert und dann in "`Ziva-Knochen"' konvertiert. \url{https://discover.therookies.co/2019/06/01/vfx-in-9-steps/}
\end{itemize}

%---------------------------
\section{ZSpheres in Zbrush}

\begin{itemize}
 \item \url{http://docs.pixologic.com/user-guide/3d-modeling/modeling-basics/creating-meshes/zspheres/},\\ Beispielvideo: \url{https://www.youtube.com/watch?v=Wl0XK6ggUOA}
 \item Möglichkeit ein "`Skelett"' aus Kugeln zu erstellen. Definiert aber eher die grobe Außenhaut mit Zusatzinformationen dazu wo die Gelenke sind.
\end{itemize}

%----------------------
\section{3DS MAX Biped}

\begin{itemize}
 \item \url{https://knowledge.autodesk.com/support/3ds-max/learn-explore/caas/CloudHelp/cloudhelp/2019/ENU/3DSMax-Character-Animation/files/GUID-2F6BC5D1-DD45-4C2E-AC3A-D8C6E0F5DEB1-htm.html}
 \item Möglichkeit Skelett in einen fertig modellierten Körper einzupassen. 
 \item Skelette sind schon vorgefertigt.
 \item v.a. für menschliche Skelette, aber auch (limitiert) anpassbar auf Tiere
\end{itemize}

%---------------------------------
\section{Forensik und Archäologie}

\begin{itemize}
 \item forensische Gesichstrekonstruktion ist spezialisiert auf Menschen und verwendet Zusatzinformationen wie Stockfotos von Gesichtsmerkmalen (\url{https://en.wikipedia.org/wiki/Forensic_facial_reconstruction})
 \item Rekonstruktion von Tieren in der Archäologie anhand des Skeletts v.a. durch Künstler (?)
\end{itemize}


%----------------------
\section{No Man's Sky}

\begin{itemize}
 \item Webseite \cite{NoMansSky}
 \item "`For creatures, basic templates of creatures that exist on the Earth were created and then manipulated by the system, changing everything from height, weight, bone density, voice pitch, what it eats, and its behaviors, even creating variation within the species."' (\url{https://nomanssky.fandom.com/wiki/Biology})
 \item "`Creatures were often generated by mixing and matching random parts from a library, and then adjusting the underlying skeleton so that the creature appeared realistic; a creature with a tiny body could not support a giant head, for example."' (\url{https://en.wikipedia.org/wiki/Development_of_No_Man\%27s_Sky})
 \item Zunächst Generierung von äußerem 3D-Modell, dann Anpassung der Knochen.
\end{itemize}

%---------------------
\section{Avatar movie}

\begin{itemize}
 \item Direhorse (Schreckenspferd) hat sechs Beine (vier vorne, zwei hinten).\\
 Bilder: \url{https://james-camerons-avatar.fandom.com/wiki/Gallery:_Pandoran_Creatures?file=Muscle.jpg}, \url{https://james-camerons-avatar.fandom.com/wiki/Gallery:_Pandoran_Creatures?file=Pandora_ROVR_Direhorse.png}
 \item Prolemuris hat Arme, mit einem Oberarm, aber zwei Unterarmen.\\
 Bild: \url{https://james-camerons-avatar.fandom.com/wiki/Gallery:_Pandoran_Creatures}
\end{itemize}


%-------------------
\section{Mythologie}

\begin{itemize}
 \item Zentauren haben Körper wie ein Pferd, aber statt einem Hals setzt ein menschlicher Oberkörper auf Schultergürtel auf.\\
 \url{https://tvtropes.org/pmwiki/pmwiki.php/Main/OurCentaursAreDifferent?from=Main.CentauroidForm}
 \item Pegasus hat den Körper eines Pferdes + Flügel, die zusätzlich zu den Vorderbeinen an Schultergürtel ansetzen\\
 \url{https://tvtropes.org/pmwiki/pmwiki.php/Main/Pegasus}
\end{itemize}





%-----------------
%-----------------
\chapter{Biologie}

\begin{itemize}
 \item "`Wirbeltiere (Vertebrata) [\dots] Von vielen Zoologen wird heute der Begriff Schädeltiere (Craniota) für dieses Taxon bevorzugt. Diese Auffassung berücksichtigt, dass die Rundmäuler, wie auch einige andere Wirbeltiere, als Achsenskelett keine Wirbelsäule, sondern eine Chorda dorsalis haben. Doch allen Wirbeltieren gemein ist ein verknöcherter oder knorpeliger Schädel; sein Vorhandensein gehört somit zu den gemeinsam abgeleiteten Merkmalen (Synapomorphien) dieser Chordaten-Gruppe."' (\url{https://de.wikipedia.org/wiki/Wirbeltiere}) $\rightarrow$ Beschränkung auf Schädeltiere mit Wirbelsäule
 \item "`Dem Skelett der Wirbeltiere sind viele Gemeinsamkeiten ansehbar, trotzdem unterscheidet es sich, je nach Lebensraum und Anforderungen, teilweise erheblich. Mit diesen Gemeinsamkeiten und Unterschieden beschäftigt sich die Vergleichende Anatomie."' (\url{https://de.wikipedia.org/wiki/Skelett#Wirbeltiere}) Notizen zu \cite{Vergleichende_Anatomie} siehe Anhang \ref{appendix_vergleichende_anatomie}.
 \item Das Skelett eines Wirbeltiers ist nicht unbedingt zusammenhängend.
 
 \item "`Säugetiere haben in der Regel sieben Halswirbel."' Bei Wirbeltieren kann die Anzahl aber zwischen $6$ und $31$ variieren. Vögel haben zwischen $10$ und $31$ und zwei Tiere haben $6$ Wirbel. (\url{https://de.wikipedia.org/wiki/Halswirbel}, weitere Quelle zu Wirbelanzahl bei Säugetieren: \url{https://archive.org/details/bub_gb_4JFmAAAAMAAJ/page/n35/mode/2up}, S.10)
    \begin{itemize}
     \item Halswirbel: 7 oder, falls Flügel 10 bis 30
     \item Wirbel auf vorderer Hälfte der Wirbelsäule: 15 (Brustwirbel)
     \item Wirbel auf hinterer Hälfte der Wirbelsäule: 10 (Lenden- + Kreuzwirbel)
     \item Schwanzwirbel: 5 bis 20
    \end{itemize}

 
 \item Form der Wirbelsäule siehe \url{https://de.wikipedia.org/wiki/Wirbels\%C3\%A4ule}
 
 \item Schultergürtel: Das Schlüsselbein ist bei den meisten Wirbeltieren zurückgebildet, bei manchen gar nicht vorhanden. Das Rabenbein ist bei vielen Wirbeltieren zu einem Fortsatz am Schulterblatt zurückgebildet. Das Schulterblatt ist meistens vorhanden. $\rightarrow$ Nur das Schulterblatt wird modelliert.\\
 \url{https://de.wikipedia.org/wiki/Schulterg\%C3\%BCrtel}
\end{itemize}
