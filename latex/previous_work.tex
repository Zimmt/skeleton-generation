%---------------------------
%---------------------------
\chapter{Bisherige Arbeiten}


%--------------------
\section{3D-Editoren}

Aufbau der erzeugten / zu erzeugenden Skelette mit Aufbau von Skeletten, die für Animation verwendet werden, vergleichen (realistischer,\dots)

%- - - - - - -
\subsection{Ziva}

\begin{itemize}
 \item Ziva VFX Maya Plugin zur Erstellung von Charakteren und Simulation von biomechanischen Bewegungen \url{https://zivadynamics.com/}
 \item Charaktererstellung in Ziva beginnt mit der Modellierung des Skeletts. Knochen mit Animationen werden als Alembic-Datei gespeichert und dann in "`Ziva-Knochen"' konvertiert. \url{https://discover.therookies.co/2019/06/01/vfx-in-9-steps/}
\end{itemize}

%- - - - - - - - - - - - - -
\subsection{ZSpheres in Zbrush}

\begin{itemize}
 \item \url{http://docs.pixologic.com/user-guide/3d-modeling/modeling-basics/creating-meshes/zspheres/},\\ Beispielvideo: \url{https://www.youtube.com/watch?v=Wl0XK6ggUOA}
 \item Möglichkeit ein "`Skelett"' aus Kugeln zu erstellen. Definiert aber eher die grobe Außenhaut mit Zusatzinformationen dazu wo die Gelenke sind.
\end{itemize}

%- - - - - - - - - - - 
\subsection{3DS MAX Biped}

\begin{itemize}
 \item \url{https://knowledge.autodesk.com/support/3ds-max/learn-explore/caas/CloudHelp/cloudhelp/2019/ENU/3DSMax-Character-Animation/files/GUID-2F6BC5D1-DD45-4C2E-AC3A-D8C6E0F5DEB1-htm.html}
 \item Möglichkeit Skelett in einen fertig modellierten Körper einzupassen. 
 \item Skelette sind schon vorgefertigt.
 \item v.a. für menschliche Skelette, aber auch (limitiert) anpassbar auf Tiere
\end{itemize}


%---------------------------------
\section{Forensik und Archäologie}

\begin{itemize}
 \item forensische Gesichstrekonstruktion ist spezialisiert auf Menschen und verwendet Zusatzinformationen wie Stockfotos von Gesichtsmerkmalen (\url{https://en.wikipedia.org/wiki/Forensic_facial_reconstruction})
 \item Rekonstruktion von Tieren in der Archäologie anhand des Skeletts v.a. durch Künstler (?)
\end{itemize}

%--------------------------------
\section{Computerspiele und Filme}

%- - - - - - - - - - -
\subsection{No Man's Sky}

\begin{itemize}
 \item Webseite \cite{NoMansSky}
 \item "`For creatures, basic templates of creatures that exist on the Earth were created and then manipulated by the system, changing everything from height, weight, bone density, voice pitch, what it eats, and its behaviors, even creating variation within the species."' (\url{https://nomanssky.fandom.com/wiki/Biology})
 \item "`Creatures were often generated by mixing and matching random parts from a library, and then adjusting the underlying skeleton so that the creature appeared realistic; a creature with a tiny body could not support a giant head, for example."' (\url{https://en.wikipedia.org/wiki/Development_of_No_Man\%27s_Sky})
 \item Zunächst Generierung von äußerem 3D-Modell, dann Anpassung der Knochen.
 \item create creatures from DB (Ribeiro et al 2003) (hat das was mit no mans sky zu tun?)
\end{itemize}

%- - - - - - - - - - -
\subsection{Film "`Avatar"'}

\begin{itemize}
 \item Direhorse (Schreckenspferd) hat sechs Beine (vier vorne, zwei hinten).\\
 Bilder: \url{https://james-camerons-avatar.fandom.com/wiki/Gallery:_Pandoran_Creatures?file=Muscle.jpg}, \url{https://james-camerons-avatar.fandom.com/wiki/Gallery:_Pandoran_Creatures?file=Pandora_ROVR_Direhorse.png}
 \item Prolemuris hat Arme, mit einem Oberarm, aber zwei Unterarmen.\\
 Bild: \url{https://james-camerons-avatar.fandom.com/wiki/Gallery:_Pandoran_Creatures}
\end{itemize}

