%------------
%------------
\chapter{PCA}

 Eine \emph{Principal Component Analysis} (PCA) oder auch Hauptkomponentenanalyse auf einer Menge von Beispielen findet diejenigen Dimensionen, in denen sich die Beispiele am meisten unterscheiden, die "`principle components"'. Das Ziel dabei ist in den meisten Fällen die Dimensionalität der Daten zu reduzieren. Das wird dadurch erreicht, dass die Eingabedaten nur noch durch die "`primary components"' dargestellt werden, da die anderen Dimensionen optimalerweise nur einen marginalen Einfluss auf die Daten haben.
 
 Das Ziel hier, war vor allem die Position und Krümmung der Wirbelsäule bei unterschiedlichen Tieren zu untersuchen, da diese sehr charakteristische Merkmale eines Skeletts sind. Aus dem gleichen Grund soll der Algorithmus zur Generierung der Skelette mit der Generierung der Wirbelsäule starten und daraus dann den Rest der Knochen "`wachsen"' lassen.
 Die Hoffnung hier ist auch, dass dadurch Skelette generiert werden, die ausbalanciert wirken. Es soll nichts generiert werden, das aussieht, als würde es sofort umfallen. \todo{Hat das funktioniert?}
 
 Eine PCA, die als Eingabe viele Beispiele für Wirbelsäulen und ein paar weiteren Daten zum zugehörigen Tier bekommt, ist hier sehr hilfreich, da sie Zusammenhänge für die Positionierung der Wirbelsäule liefert.
 
 
 %-----------------------
 \section{Funktionsweise}
 
 \todo{PCA Quelle}
 
 Die Eingabe für eine PCA ist eine Menge von Punkten. Diese Punkte repräsentieren einzelne Instanzen dessen, was untersucht werden soll, hier ein Skelett. Für jede Eigenschaft des Skeletts hat der Punkt eine Dimension. Gegeben ist also eine Menge von Punkten $P$ mit jeweils $d$ Dimensionen. Es wird angenommen, dass die Punkte in jeder Dimension normal-/gaußverteilt sind. Dann bilden die Punkte im $d$-dimensionalen Raum einen Ellipsoid.
 
 Das Ziel der PCA ist herauszufinden wo die Achsen des Ellipsoids liegen, also wie die Eingabedimensionen miteinander korreliert sind. Interessant sind dabei die Achsen in deren Richtung die Daten die größe Streuung aufweisen.
 
 Um diese Achsen zu Berechnen wird zunächst die Kovarianzmatrix aufgestellt, deren Einträge die Kovarianz zwischen den verschiedenen Achsen beschreibt.
 Die Eigenvektoren dieser Kovarianzmatrix sind dann die Achsen des Ellipsoids und die zugehörigen Eigenwerte geben an wie groß die Varianz in dieser Richtung ist, also weit ausgedehnt das Ellipsoid in dieser Richtung ist. Der Mittelpunkt des Ellipsoids ist der Mittelwert der Daten.
 
 Will man nun herausfinden was die Haupteigenschaften eines Datenpunktes sind, stellt man ihn im neuen Koordinatensystem des Ellipsoids dar, also als gewichtete Summe der Eigenvektoren, und betrachtet dann nur die Dimensionen mit den größten Eigenwerten. Dazu zieht man zunächst den Mittelwert vom Datenpunkt ab und multipliziert ihn dann mit der transponierten Basiswechselmatrix, also der Matrix, in der in den Zeilen die Eigenvektoren stehen.


 %---------------------- 
 \section{Datenerhebung}
 
 Die konkret erhobenen Beispiele sind vor allem der Datenlage \bzw der zugänglichen Quellen geschuldet. Trotzdem wurde darauf geachtet möglichst viele unterschiedliche Tierarten mit viel Variation in den erhobenen Merkmalen zu finden.
 
 Viele Beispiele stammen Zoologiebüchern, in denen sie als Beispiele für bestimmte Erklärungen angegeben waren (Bildquellen siehe Anhang \ref{appendix_pca_skeletons}). Dem ist auch geschuldet, dass recht viele Dinosaurierskelette dabei sind. Denn von anderen Tieren gibt es als alternative Darstellung eine Außenansicht des lebenden Tieres. Das geht bei ausgestorbenen Tieren im Allgemeinen nicht.
 
 Die Merkmale, die zur Datenerhebung ausgesucht wurden, sind charakteristisch für ein Skelett, sie tragen also viel zum Gesamteindruck bei. Das sind vor allem der Verlauf der Wirbelsäule und der Aufbau der Extremitäten.
  
 Eingeschränkt wurde die Erhebung natürlich auch durch die begrenzte Datenlage. Am einfachsten zu bekommen sind 2D-Bilder mit Seitenansichten von Skeletten. Das schließt Merkmale aus, die Tiefeninformationen benötigen, \zb den Abstand der Füße oder die Winkel der Gelenke an den Beinen. Auch Informationen zu sehr kleinen Knochen, wie Handwurzelknochen oder die unterschiedlichen Fingerknochen, sind schwierig zu bekommen, da sie teilweise schwierig zu Erkennen und zu Markieren sind. Deshalb haben wir die Erhebung auf folgende Daten eingeschränkt:
  
 \begin{itemize}
  \item Ein Bild mit der Seitenansicht des Skeletts.
  Darin wurde die Lage der Wirbelsäule und die Länge der Knochen der Vorder- und Hintergliedmaßen markiert, falls vorhanden.\\
  (Die Quellen hierfür sind in Anhang \ref{appendix_pca_skeletons} zu finden.)
  
  \item Die Tierklasse, also ob das Tier ein Fisch, ein Amphib, ein Reptil oder ein Säugetier ist. Diese Daten lassen sich nicht auf einer kontinuierlichen Skala abbilden und sind deshalb nicht als Eingabedimension für die PCA geeignet. Sie wurden trotzdem erhoben, da sie für eine anderweitige Auswertung hilfreich sein könnten.
  
  \item Ob Flügel vorhanden sind.
  
  \item Die Anzahl der Beine mit Bodenkontakt geteilt durch zwei. (Die Skalierung mit $2$ ist nicht relevant für die PCA, soll aber repräsentieren, dass ein Tier immer eine gerade Anzahl an Extremitäten besitzt.)
  
  \item Das ungefähre Gewicht eines ausgewachsenen Exemplars in Kilogramm. Hier wurde oft das maximale Gewicht verwendet, da keine Angaben zum Durchschnittsgewicht zu finden waren. Teilweise gibt es auch verschiedene (Unter-)Arten, die unterschiedlich schwer werden können, aber, bei der Auflösung der hier erhobenen Daten, das gleiche Gewicht haben. In diesem Fall wurde ein beliebiger Wert gewählt, der zwischen dem Gewicht der leichtesten und dem der schwersten (Unter-)Art liegt. (Die Quellen hierfür sind im Anhang \ref{appendix_pca_weight} zu finden.)
 \end{itemize}

 Die Bilder der Skelette wurden folgendermaßen für die Datenerhebung vorbereitet:
 
 \begin{enumerate}
  \item Zuschneiden des Bildes, so dass möglichst nur das Skelett mit wenig Rand außen herum zu sehen ist.
  \item Einfügen in eine $1000 \times 1000$ Pixel große Bildumgebung.
  \item Verschieben innerhalb der Bildumgebung an den unteren Rand und horizontal in die Mitte.
 \end{enumerate}

 Ist das geschehen kann die Lage der Wirbelsäule und die Länge der Knochen der Extremitäten annotiert werden.
 
 Die Lage der Wirbelsäule wird durch drei kubische Bézierkurven erfasst. Jeweils einer für Hals, Rücken und Schwanz. Hals und Rücken teilen sich einen Punkt und Rücken und Schwanz. Die Punkte an denen sie ineinander übergehen sind der Schultergürtel und der Beckengürtel.
 Das sind die ersten $20$ Eingabedimensionen für die PCA ($10$ zweidimensionale Punkte).
 
 Zusätzlich wird jeweils durch eine Gerade im Bild die Länge des Ober- und Unterarms, der Hand, des Ober- und Unterschenkels und des Fußes eingetragen, falls vorhanden. Die Bezeichnung der Extremität als Arm oder Bein ist nur zur Unterschiedung zwischen Vorder- und Hinterextremitäten gedacht. Sie hat in keiner Weise etwas mit der Funktion der Gliedmaßen zu tun.
 
 Zusätzlich zum Bild gibt es noch eine Textdatei, in der die restlichen Daten erfasst werden.

 Alle erfassten Dimensionen werden vor der Weiterverarbeitung durch die PCA auf das Intervall $[0, 1]$ skaliert, so dass jede Dimension den gleichen Einfluss auf das Ergebnis hat. 
 Um das zu erreichen werden alle Daten einer Dimension durch den maximal möglichen Wert geteilt.
 
 \begin{itemize}
  \item Koordinaten oder Längen im Bild liegen im Intervall $[0, 1000]$, da sie in Pixeln dargestellt werden und das Bild eine Größe von $1000 \times 1000$ Pixel hat. Deshalb werden sie mit $1000$ skaliert. Bei Längen wären theoretisch auch Werte $> 1000$ möglich. Solche Längen wären aber unrealistisch und werden deshalb ignoriert.
  
  \item Die Angabe, ob das Tier Flügel hat oder nicht, wird mit $0$ oder $1$ dargestellt, muss also nicht skaliert werden.
  
  \item Die Anzahl der Extremitäten geteilt durch $2$ ist maximal $2$, wird also mit $2$ skaliert.
  
  \item Das Gewichtsdaten sind nicht normalverteilt. Stellt man sie aber mit logarithmischer Skala dar, sind sie das (siehe unten, "`Analyse der Eingabedaten"'). Deshalb wird das Gewicht $w$ für die PCA wie folgt umgerechnet: $\frac{\mathrm{log}(w+1)}{\mathrm{log}(\mathrm{max}+1)}$. Das schwerste Wirbeltier ist der Blauwal mit bis zu 120 Tonnen (siehe \ref{appendix_pca_weight}). Deshalb ist hier max $= 120.000$.
\end{itemize}

 Generell bewirkt die Skalierung einer Dimension eine Gewichtung. Denn durch eine Skalierung ändert sich auch die (Co-)Varianz und somit auch die Kovarianzmatrix. Seien beispielsweise $s,t \in \mathbb{R}$, dann bewirkt eine Skalierung mit $s$ in Dimension $x$ und eine Skalierung mit $t$ in Dimension $y$ eine Skalierung von $s \cdot t$ der Kovarianz Cov$(x,y)$ von $x$ mit $y$, da $\mathrm{Cov}(sx, ty) = (sx - s\mu_x) (ty - t\mu_y) = st \cdot \mathrm{Cov}(x,y)$, mit Erwartungswert $\mu_i$ in Dimension $i$.
 
 Die Daten einer Dimension werden nicht mit dem maximal angenommenen Wert skaliert, sondern mit dem maximal möglichen. Das bedeutet bei Koordinaten und Längen im Bild, dass sie größeren \bzw kleineren Einfluss haben, je nach dem wieviel Raum sie im Bild einnehmen, also wieviel sie zum Gesamteindruck des Tieres beitragen.
 
 Eine andere Skalierung, die manchen Dimensionen mehr oder weniger Gewicht gibt, wäre sicherlich auch möglich gewesen. Da es aber keine besonderen Gründe für eine andere Skalierung gab, wurde diese gewählt.
 
 %- - - - - - - - - - - - - -
 \subsection{Schwierigkeiten}
 
 \begin{itemize}
  \item Bei Fischen ist nicht klar wo Rücken in Schwanz übergeht, da der Beckengürtel sich teilweise beim Kopf befindet oder auch gar nicht vorhanden ist. Bei der Datenerhebung wurde der Übergang ungefähr bei der Rücken- oder der Afterflosse festgelegt, da dies relativ gut zum Algorithmus passt.
  
  \item Hals und Schwanz von manchen Tieren ist mit einer kubischen Bézierkurve nicht darstellbar. Das ist unter den verwendeten Beispielen der Hals von Ichthyornis und vom Schwan und der Schwanz vom Ichthyosaurus und vom Koboldmaki. In diesem Fällen wurde versucht die Form möglichst gut anzunähern oder Fortsätze (wie am Schwanz vom Ichthyosaurus) die im Algorithmus wahrscheinlich sowieso nicht abgebildet werden, einfach wegzulassen.
  
  \item Die Schwanzposition bei Tieren mit sehr langen Schwänzen ist auf den Bildern relativ beliebig. Hier wurde versucht den Schwanz möglichst gerade nach hinten fortzusetzen, auch wenn er auf dem Bild irgendwie eingerollt ist.
  
 \end{itemize}
 
 %---------------------------------
 \section{Analyse der Eingabedaten}
 
 \todo{PCA Daten aktualisieren, Plots einfügen}
 
 \begin{itemize}
  \item Test auf Gaußverteilung (Q Q Plot) \todo{verstehen, programmieren}
  \item min/max Abstand zum Mittelwert
  \item mit/ohne Flügel: Anzahl, Flügelcluster
  \item Gewicht logarithmisch wegen Verteilung, Gewicht ist dann auch nicht mehr kleiner null (egal wie logarithmisch, ist nur Skalierung)
 \end{itemize}
 
 
 %- - - - - - - - - - - - - - - - - -
 \section{Analyse der PCA-Ergebnisse}

 \begin{itemize}
  \item Eigenvektoren/-werte
  \item Rekonstruktion der Tiere nur anhand der größten Eigenwerte (funktioniert so mittel gut\dots; gute und schlechte Beispiele zeigen)
  \item Möglichkeit Gewicht, Anzahl Beine mit Bodenkontakt, Flügel wegzulassen. Ergebnisse nicht total unterschiedlich (Unterschiede vor allem bei Extrema wie sehr schweren Tieren und/oder Fischen)\\
  keine Möglichkeit zu Bewerten was besser ist (Bewertung nach Abweichung der Rekonstruktion aus größten Eigenvektoren zu Original nicht sinnvoll, da Dimensionen unterschiedlich oder nicht vorhanden). Mehr Daten für Algorithmus hilfreich, also Daten behalten.
  \item Möglichkeit PCA-Daten nach Flügel/nicht Flügel aufzuteilen (führt zu besserer Anpassung an Eingabedaten, aber keine Approximation an Datenpunkte zwischen Clustern) \todo{man könnte auch genauso gut anhand von anderen Merkmalen aufteilen}
 \end{itemize}
 


