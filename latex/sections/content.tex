
\chapter{Bisherige Arbeiten}

\section{Skelett der Wirbeltiere}

\begin{itemize}
 \item "`Dem Skelett der Wirbeltiere sind viele Gemeinsamkeiten ansehbar, trotzdem unterscheidet es sich, je nach Lebensraum und Anforderungen, teilweise erheblich. Mit diesen Gemeinsamkeiten und Unterschieden beschäftigt sich die Vergleichende Anatomie."' (\url{https://de.wikipedia.org/wiki/Skelett#Wirbeltiere}) $\rightarrow$ über vergleichende Anatomie informieren
 \item "`Von vielen Zoologen wird heute der Begriff Schädeltiere (Craniota) für dieses Taxon bevorzugt. Diese Auffassung berücksichtigt, dass die Rundmäuler, wie auch einige andere Wirbeltiere, als Achsenskelett keine Wirbelsäule, sondern eine Chorda dorsalis haben. Doch allen Wirbeltieren gemein ist ein verknöcherter oder knorpeliger Schädel; sein Vorhandensein gehört somit zu den gemeinsam abgeleiteten Merkmalen (Synapomorphien) dieser Chordaten-Gruppe."' (\url{https://de.wikipedia.org/wiki/Wirbeltiere}) $\rightarrow$ Beschränkung auf Schädeltiere mit Wirbelsäule
\end{itemize}


%----------------------
\section{No Man's Sky}

\begin{itemize}
 \item Webseite \cite{NoMansSky}
 \item "`For creatures, basic templates of creatures that exist on the Earth were created and then manipulated by the system, changing everything from height, weight, bone density, voice pitch, what it eats, and its behaviors, even creating variation within the species."' (\url{https://nomanssky.fandom.com/wiki/Biology})
 \item "`Creatures were often generated by mixing and matching random parts from a library, and then adjusting the underlying skeleton so that the creature appeared realistic; a creature with a tiny body could not support a giant head, for example."' (\url{https://en.wikipedia.org/wiki/Development_of_No_Man\%27s_Sky})
 \item Zunächst Generierung von äußerem 3D-Modell dann Anpassung der Knochen. Hier zunächst Generierung eines "`biologisch sinnvollen"' Skeletts, dann Außenhülle.
\end{itemize}


\chapter{Idee}

\begin{enumerate}
 \item Erzeugung eines Wirbeltierskeletts
 \item Erzeugung von Muskeln (optional?)
 \item Erzeugung von Haut (einfache Hülle um das vorhandene Modell) (optional)
\end{enumerate}

%---------------
\section{Skelett}

\begin{itemize}
 \item Iterative Erzeugung eines Skeletts durch eine probabilistische kontextfreie (?) Grammatik, die so erweitert ist, dass sie nicht ein einfaches Wort erzeugt, sondern einen Baum von Zeichen. Verwendung von paramterischen L-Systemen \cite{Paramteric_L-Systems} könnte sinnvoll sein.
 \item Der Zufall ist eingeschränkt durch Eingabeparameter oder Benutzereingabe.
 \item Ein Skelett besteht aus Knochen, Gelenken und deren (relativen) Positionen und Orientierungen.
 \item Ein Knochen ist im einfachsten Fall ein Zylinder (Strecke) mit Länge und Radius, kann aber auch eine konvexe Kurve mit einem Radius sein.
 \item Ein Gelenk verbindet hat keine Ausdehnung (?). Es ist das Verbindungsstück zwischen zwei oder mehr Knochen und legt fest wie die Knochen sich relativ zueinander bewegen können. Werden mehr als zwei Knochen verbunden ist es einfach eine feste Verbindung.
\end{itemize}

%----------------
\section{Muskeln}

\begin{itemize}
 \item Die "`Hauptmuskeln"' verlaufen bei Wirbeltieren wahrscheinlich ähnlich, relativ zu den Knochen.
 \item Knochen/Gelenke bekommen Zusatzattribute für Start- und Zielpunkte der Muskeln.
 \item Muskeln haben eine "`Dicke"' und aus Start- und Zielpunkt muss Kurve des Muskels generiert werden.
 \item Wie wird genau Form festgelegt? Muskeln irgendwie auf ihre "`Dicke"' aufblähen + Interaktion mit vorhandenen Elementen (andere Muskeln und Knochen)
\end{itemize}

%-------------
\section{Haut}

\begin{itemize}
 \item Was für Algorithmen gibt es, die zu einem vorhandenen 3D-Modell eine Hülle mit gewissen Eigenschaften generieren? \\
 es gibt eine solche Funktion z.B. in AutoCAD \url{https://knowledge.autodesk.com/de/support/autocad/learn-explore/caas/CloudHelp/cloudhelp/2016/DEU/AutoCAD-Core/files/GUID-B7F99810-765E-4E7E-ABDD-275C64147CCC-htm.html}
 \item Muskeln sind dafür nicht unbedingt nötig, aber wahrscheinlich hilfreich.
\end{itemize}


\chapter{Technische Umsetzung Skelett}

\begin{itemize}
 \item Repräsentation des Zustands als Baum von einzelnen Zeichen (terminale sowie nichtterminale).
 \item Programmiersprache: Rust?
 \item Übersetzung in ein 3D-Modell
    \begin{itemize}
     \item Welches Dateiformat? (siehe unten)
     \item Geht das mit Rust oder braucht man noch einen Zwischenschritt? Oder doch besser andere Programmiersprache?
    \end{itemize}
 \item Wie sieht Benutzereingabe aus?
\end{itemize}




%---------------------
\section{Dateiformate}

\begin{itemize}
 \item Welches Format wird am meisten Verwendet? $\rightarrow$ fbx?
 \item Einfachstes Format (aber auch mit kleinstem Funktionsumfang) obj
 \item Welches Format ist für eine interaktive Anwendung sinnvoll?
\end{itemize}

\subsection{OBJ}

\begin{itemize}
 \item Beschreibung des Formats: \url{https://www.fileformat.info/format/wavefrontobj/egff.htm}
 \item Erzeugung mit Rust: obj\_exporter \url{https://docs.rs/obj-exporter/0.2.0/obj_exporter/index.html}
\end{itemize}

\subsection{FBX}

\begin{itemize}
 \item Verwendung am besten über Autodesk FBX SDK für C++. 
 \item Dokumentation: \url{http://help.autodesk.com/view/FBX/2019/ENU/}
 \item Es gibt auch fbxcel, eine FBX library für Rust. Ist aber relativ low level und nicht ganz offensichtlich wie zu verwenden.
\end{itemize}

