
\chapter{Bisherige Arbeiten}

%-------------
\section{Ziva}

\begin{itemize}
 \item Ziva VFX Maya Plugin zur Erstellung von Charakteren und Simulation von biomechanischen Bewegungen \url{https://zivadynamics.com/}
 \item Charaktererstellung in Ziva beginnt mit der Modellierung des Skeletts. Knochen mit Animationen werden als Alembic-Datei gespeichert und dann in "`Ziva-Knochen"' konvertiert. \url{https://discover.therookies.co/2019/06/01/vfx-in-9-steps/}
\end{itemize}

%---------------------------
\section{ZSpheres in Zbrush}

\begin{itemize}
 \item \url{http://docs.pixologic.com/user-guide/3d-modeling/modeling-basics/creating-meshes/zspheres/},\\ Beispielvideo: \url{https://www.youtube.com/watch?v=Wl0XK6ggUOA}
 \item Möglichkeit ein "`Skelett"' aus Kugeln zu erstellen. Definiert aber eher die grobe Außenhaut mit Zusatzinformationen dazu wo die Gelenke sind.
\end{itemize}

%----------------------
\section{3DS MAX Biped}

\begin{itemize}
 \item \url{https://knowledge.autodesk.com/support/3ds-max/learn-explore/caas/CloudHelp/cloudhelp/2019/ENU/3DSMax-Character-Animation/files/GUID-2F6BC5D1-DD45-4C2E-AC3A-D8C6E0F5DEB1-htm.html}
 \item Möglichkeit Skelett in einen fertig modellierten Körper einzupassen. 
 \item Skelette sind schon vorgefertigt.
 \item v.a. für menschliche Skelette, aber auch (limitiert) anpassbar auf Tiere
\end{itemize}

%---------------------------------
\section{Forensik und Archäologie}

\begin{itemize}
 \item forensische Gesichstrekonstruktion ist spezialisiert auf Menschen und verwendet Zusatzinformationen wie Stockfotos von Gesichtsmerkmalen (\url{https://en.wikipedia.org/wiki/Forensic_facial_reconstruction})
 \item Rekonstruktion von Tieren in der Archäologie anhand des Skeletts v.a. durch Künstler (?)
\end{itemize}


%----------------------
\section{No Man's Sky}

\begin{itemize}
 \item Webseite \cite{NoMansSky}
 \item "`For creatures, basic templates of creatures that exist on the Earth were created and then manipulated by the system, changing everything from height, weight, bone density, voice pitch, what it eats, and its behaviors, even creating variation within the species."' (\url{https://nomanssky.fandom.com/wiki/Biology})
 \item "`Creatures were often generated by mixing and matching random parts from a library, and then adjusting the underlying skeleton so that the creature appeared realistic; a creature with a tiny body could not support a giant head, for example."' (\url{https://en.wikipedia.org/wiki/Development_of_No_Man\%27s_Sky})
 \item Zunächst Generierung von äußerem 3D-Modell, dann Anpassung der Knochen.
\end{itemize}

%-----------------
%-----------------
\chapter{Biologie}

\begin{itemize}
 \item "`Wirbeltiere (Vertebrata) [\dots] Von vielen Zoologen wird heute der Begriff Schädeltiere (Craniota) für dieses Taxon bevorzugt. Diese Auffassung berücksichtigt, dass die Rundmäuler, wie auch einige andere Wirbeltiere, als Achsenskelett keine Wirbelsäule, sondern eine Chorda dorsalis haben. Doch allen Wirbeltieren gemein ist ein verknöcherter oder knorpeliger Schädel; sein Vorhandensein gehört somit zu den gemeinsam abgeleiteten Merkmalen (Synapomorphien) dieser Chordaten-Gruppe."' (\url{https://de.wikipedia.org/wiki/Wirbeltiere}) $\rightarrow$ Beschränkung auf Schädeltiere mit Wirbelsäule
 \item "`Dem Skelett der Wirbeltiere sind viele Gemeinsamkeiten ansehbar, trotzdem unterscheidet es sich, je nach Lebensraum und Anforderungen, teilweise erheblich. Mit diesen Gemeinsamkeiten und Unterschieden beschäftigt sich die Vergleichende Anatomie."' (\url{https://de.wikipedia.org/wiki/Skelett#Wirbeltiere}) Notizen zu \cite{Vergleichende_Anatomie} siehe Anhang \ref{appendix_vergleichende_anatomie}.
 \item Das Skelett eines Wirbeltiers ist nicht unbedingt zusammenhängend.
\end{itemize}

%-------------
%-------------
\chapter{Idee}

\begin{enumerate}
 \item Erzeugung eines Wirbeltierskeletts
 \item Erzeugung von Muskeln (future work)
 \item Erzeugung von Haut (future work / kurz ausprobieren)
 \item Erzeugung von vielen Skelettvarianten bei Eingabe eines Skeletts (nur wenn es relativ leicht möglich ist)
\end{enumerate}

%---------------
\section{Skelett}

\begin{itemize}
 \item Iterative Erzeugung eines Skeletts durch eine probabilistische kontextfreie (?) Grammatik, die so erweitert ist, dass sie nicht ein einfaches Wort erzeugt, sondern einen Baum von Zeichen (nötig für Extremitäten). Verwendung von paramterischen L-Systemen \cite{Paramteric_L-Systems} könnte sinnvoll sein.
 \item Regeln sind nicht wirklich eine Grammatik, da fast jedes nichtterminale Literal nur einmal vorkommt, wenn es für jedes Körperteil andere Regeln gibt. Oder ist es möglich so zu abstrahieren, dass z.B. Arme und Beine den gleichen/ähnlichen Regeln unterliegen? Ist das sinnvoll? \todo{Abstraktionsgrad, Art der Regeln}\\%TODO
 Außerdem ist das Skelett nicht unbedingt zusammenhängend (siehe Biologie). $\rightarrow$ Darauf achten, dass das nicht von Algorithmus verlangt wird
 \item Regeln wichtig, die dafür sorgen, dass das Tier am Ende auch funktional ist.
 \item Darstellung des Skeletts in "`sinnvoller"' Pose.
 \item Beachte, dass Tier nicht umfällt: berechne Drehmomente um Kontaktpunkte mit Boden. Dafür sind aber Muskeln bzw.\ Masse wichtig $\rightarrow$ Schätzung der Masse anhand der Knochengröße bzw.\ Größe der Bounding Box?
 \item Der Zufall ist eingeschränkt durch Eingabeparameter oder Benutzereingabe (siehe Interaktivität). Ähnliche Parameter sollten zu ähnlichem Tier führen. Parameter könnten folgendes beschreiben:
 \begin{itemize}
  \item äußere Einflüsse (Klima, Terrain, Lebensraum,\dots)
  \item Proportionen (länge der Extremitäten, Kopfgröße,\dots)
  \item Anzahl vorhandener Gliedmaßen / Zehen / \dots
 \end{itemize}

 \item Ein Skelett besteht aus Knochen, Gelenken und deren (relativen) Positionen und Orientierungen.
 \item Ein Knochen ist im einfachsten Fall ein Zylinder (Strecke) mit Länge und Radius, kann aber auch eine konvexe Kurve mit einem Radius sein.
 \item Ein Gelenk hat keine Ausdehnung (?). Es ist das Verbindungsstück zwischen zwei oder mehr Knochen und legt fest wie die Knochen sich relativ zueinander bewegen können. Werden mehr als zwei Knochen verbunden ist es einfach eine feste Verbindung.
 \item Skelett darf nicht zu abstrakt sein, da es sonst zu wenig Informationen zum konkreten Tier liefert.
 \item Ein detailliertes Skelett ist für Wesen sinnvoll, die es so noch nicht gibt bzw. wo man keine richtige Vorstellung davon hat wie es "`funktioniert"', z.B. bei mehr Gliedmaßen.
 \item Köpfe sind kompliziert $\rightarrow$ Auswahl an Köpfen bereitstellen (evtl. leicht skalier-/verformbar oder ineinander überführbar)
\end{itemize}

%- - - - - - - - - - - - 
\subsection{Extremitäten}

\begin{itemize}
 \item \url{https://de.wikipedia.org/wiki/Extremit\%C3\%A4tenevolution}
 \item "`Die paarigen Flossen von Fischen und Extremitäten von Tetrapoden sind insofern homologe Skelettelemente, als sie bei beiden an Schulter- und Beckengürtel ansetzen und die Extremitäten aus den paarigen Flossen evolutionär hervorgegangen sind.\cite{homology} Sie unterschieden sich jedoch im Knochenaufbau und in der Embryonalentwicklung, so dass ein evolutionärer Übergang aus den Einzelelementen schwer erklärbar ist."'\\
 $\rightarrow$ einfach ignorieren und trotzdem (erstmal) so modellieren?
\end{itemize}


%----------------
\section{Muskeln}

\begin{itemize}
 \item Die "`Hauptmuskeln"' verlaufen bei Wirbeltieren wahrscheinlich ähnlich, relativ zu den Knochen. Trotzdem unterscheiden sie sich recht stark.
 \item Knochen/Gelenke bekommen Zusatzattribute für Start- und Zielpunkte der Muskeln.
 \item Muskeln haben eine "`Dicke"' und aus Start- und Zielpunkt muss Kurve des Muskels generiert werden.
 \item Wie wird die genaue Form festgelegt? Muskeln irgendwie auf ihre "`Dicke"' aufblähen + Interaktion mit vorhandenen Elementen (andere Muskeln und Knochen) $\rightarrow$ future work
\end{itemize}

%-------------
\section{Haut}

\begin{itemize}
 \item Was für Algorithmen gibt es, die zu einem vorhandenen 3D-Modell eine Hülle mit gewissen Eigenschaften generieren? \\
 es gibt eine solche Funktion z.B. in AutoCAD \url{https://knowledge.autodesk.com/de/support/autocad/learn-explore/caas/CloudHelp/cloudhelp/2016/DEU/AutoCAD-Core/files/GUID-B7F99810-765E-4E7E-ABDD-275C64147CCC-htm.html}
 \item Einfach nur eine Hülle mit gewissem Abstand sieht wahrscheinlich sehr unrealistisch aus. "`Bony Landmarks"' (Stellen an denen das Gewebe über den Knochen sehr dünn ist) könnten helfen (siehe \url{https://www.proko.com/landmarks-of-the-human-body/}) oder "`bone weights"'
\end{itemize}

%-----------------------
\section{Interaktivität}

\begin{itemize}
 \item Eine Anwendung, bei der nach Eingabe von Parametern sofort das komplette Tier generiert wird, ist weniger hilfreich als eine, bei der schrittweise Teile davon generiert werden können (und auch rückgängig gemacht werden können)
 \item Teile, die einem nicht gefallen, sollten geändert werden können
 \item Änderungen können Auswirkungen auf den Rest des Körpers haben (durch Regeln) bzw. manche Änderungen sind nicht möglich
 \item Könnte verwendet werden um schnell verschiedene Möglichkeiten zu testen
\end{itemize}

%-----------------------------
%-----------------------------
\chapter{Technische Umsetzung}

\begin{itemize}
 \item Repräsentation des Zustands als Hierarchie von einzelnen Komponenten (terminale sowie nichtterminale).
 \item Übersetzung in ein 3D-Modell: zunächst .obj, später Verwendung von OpenGL mit vertex shadern etc.\
\end{itemize}


\section{Programmiersprache}

\begin{itemize}
 \item Rust: nicht geeignet, da Datenstrukturen die zyklische Referenzen auf veränderbare Objekte verwenden nicht oder nur kompliziert umsetzbar sind.
 \item Java: scheint gut zu funktionieren. Es gibt Bibliotheken zum im-/exportieren von obj-Dateien und Unterstützung für OpenGL
\end{itemize}


%---------------------
\section{Dateiformate}

\begin{itemize}
 \item Einfachstes Format (nur für die Darstellung von 3D-Objekten ohne Zusatzinformationen): obj
 \item Erster Schritt: einfaches .obj erzeugen und mit Blender darstellen; einfach Knochen als Bounding Box darstellen
 \item Jeder Editor geht mit Muskeln und Gelenken anders um. Gibt es ein Dateiformat, das nicht speziell zu einem Editor gehört, dass Bedingungen an die Rotation von Gelenken speichern kann?
 \item Eigenes Format erzeugen? Dann bräuchte man Plugins um es in verschiedenen Editoren laden zu können. Viel verwendeter Editor: Houdini (kostenlos für Studenten aber nicht Open Source). Oder selbst darstellen (siehe Interaktivität).
\item Vorschlag von Jo: "`Memory dumps"', also direkt die structs aus dem speicher auf platte rausschreiben. Am besten wenn sie am Stueck liegen mit einem fwrite() und zurücklesen mit einem fread(). Es ist nuetzlich dazu am Anfang der Datei ein bisschen Metadaten zu speichern (magic number, version, array size etc).
\end{itemize}

%- - - - - - - - - -
\subsection{OpenSim}

\begin{itemize}
 \item \url{https://simtk-confluence.stanford.edu:8443/display/OpenSim/OpenSim+Documentation}
 \item Open Source Software Platform für die Modellierung uns Simulation von Menschen, Tieren, etc.\\
 aber vor allem gedacht zur Auswertung von experimentellen Daten
 \item Import von .obj Dateien möglich. Außerdem zusätzliche Daten wie Winkel von Gelenken über .mot oder .sto Dateien (eigenes Format von OpenSim, siehe \url{https://simtk-confluence.stanford.edu:8443/display/OpenSim/Preparing+Your+Data})
 \item Export in andere Dateiformate nicht möglich (?)
 \item für Download und Zugang zur "`Community"' Account nötig
 \item für Windows und Mac OS (Linux Support gibt es auch, ist aber schwieriger: \url{https://simtk-confluence.stanford.edu:8443/display/OpenSim/Linux+Support})
\end{itemize}


%- - - - - - - -
\subsection{OBJ}

\begin{itemize}
 \item Beschreibung des Formats: \url{https://www.fileformat.info/format/wavefrontobj/egff.htm}
 \item Erzeugung mit Rust: obj\_exporter \url{https://docs.rs/obj-exporter/0.2.0/obj_exporter/index.html}
 \item Erzeugung mit Java: javagl Obj \url{https://github.com/javagl/Obj}, unterstützt auch Umwandlung von obj-Daten in Daten, die direkt für vertex buffer objects in OpenGL verwendet werden können
 \item Reicht wahrscheinlich für die ersten Dinge aus. Finetuning wird sowieso mit anderer Software gemacht
\end{itemize}

%- - - - - - - -
\subsection{FBX}

\begin{itemize}
 \item Verwendung am besten über Autodesk FBX SDK für C++. 
 \item Dokumentation: \url{http://help.autodesk.com/view/FBX/2019/ENU/} und \url{http://docs.autodesk.com/FBX/2014/ENU/FBX-SDK-Documentation/index.html}
 \item Es gibt auch fbxcel, eine FBX library für Rust. Ist aber relativ low level und nicht ganz offensichtlich wie zu verwenden.
 \item Einschränkungen für Gelenke können in FBX nicht gespeichert werden \url{http://docs.autodesk.com/FBX/2014/ENU/FBX-SDK-Documentation/index.html?url=cpp_ref/class_fbx_constraint.html,topicNumber=cpp_ref_class_fbx_constraint_htmlc57a3f99-513a-44a0-a24f-445e9077c99f}
\end{itemize}

%- - - - - - - - - -
\subsection{Alembic}

\begin{itemize}
 \item \url{www.alembic.io}
 \item Wird u.a. dafür verwendet Knochen (+ Animationen) in Ziva zu importieren
 \item Es kann mit Python (PyAlembic) und C++ verwendet werden.\\
 PyAlembic Doku: \url{http://docs.alembic.io/python/examples.html#pyalembic-intro}\\
 C++ API Refernce (enthält sehr wenig Infos): \url{http://docs.alembic.io/reference/index.html}
 \item Für Rust gibt es keine Bibliothek (?)
\end{itemize}

%-----------------------
\section{Interaktivität}

\begin{itemize}
 \item OpenGL
 \begin {itemize}
  \item SDL + OpenGL Tutorials \\ \url{http://headerphile.com/sdl2/opengl-part-1-sdl-opengl-awesome/}, \\ \url{http://www.sdltutorials.com/sdl-opengl-tutorial-basics}
  \item Daten direkt mit OpenGL erzeugen (laden als vertex und index array)
 \end {itemize}

 \item Benutzeroberfläche
 \begin{itemize}
  \item imgui (opengl/vulcan/3D view integriert) mit Rust oder C++: \url{https://github.com/ocornut/imgui}
  
  \begin{itemize}
   \item OpenGL und Imgui für Rust: \url{https://nercury.github.io/rust/opengl/tutorial/2018/02/08/opengl-in-rust-from-scratch-00-setup.html}, \url{https://github.com/michaelfairley/rust-imgui-sdl2}
   \item es gibt Java Bindings (\url{https://github.com/ice1000/jimgui}), aber Swing ist wahrscheinlich einfacher
   \item OpenGL scene $\rightarrow$ imgui: \url{https://gamedev.stackexchange.com/questions/140693/how-can-i-render-an-opengl-scene-into-an-imgui-window}
  \end{itemize}

  
  \item Java Swing Bibliothek und JOGL (Java OpenGL Binding) (\url{http://www.jogl.info})
 \end{itemize}
\end{itemize}

%------------------------
\section{Datenstrukturen}

\begin{itemize}
 \item Nichtterminale als Bounding Box und Terminale zunächst auch. Später werden sie durch Knochenmodelle ersetzt. Für jeden möglichen Knochen wird ein Modell gespeichert. Diese Modelle werden so skaliert, dass sie einen 1x1-Würfel ausfüllen. So kann dann später leicht die Skalierung auf diesen Würfel übertragen werden.
 \item Knochen: Bounding Box bzw. Skalierung in drei Raumrichtungen, Position, Orientierungen und Name bzw. ID (wird für Ersetzung benötigt). Schätzwert für anhängende Masse nötig für Balance nötig (einfach direkt abhängig von Knochengröße?)
 \item Knochen in Hierarchie anordnen und Position und Rotation relativ zu Elternelement angeben, da damit Erzeugung von Kindelementen einfacher $\rightarrow$ verwende homogene Koordinaten.
 \item Gelenke: Position, verbundene Knochen, Bewegungseinschränkungen für alle drei Richtungen; \todo{Bewegungseinschränkungen für bestimmtes Gelenk bei jedem Tier gleich?}, %TODO 
 Wenn Skelett nur dargestellt werden soll und nicht bewegt: \todo{Bewegungseinschränkungen überhaupt nötig}, %TODO
 kein 3D-Modell nötig (?)
 \item Erzeugung von Kindelementen nicht nur von direktem Elternteil abhängig $\rightarrow$ Abhängigkeiten zwischen entfernten Teilen nötig; auch für Balance nützlich. "`Kommunikation"' über Überobjekt, das auch alle Einzelteile speichert.
 \item Nichtterminale speichern: Bounding Box, Gewicht [min, max] (falls Einschränkungen vorhanden)  
 \item Berechnung der Blance: Drehmomente der Knochen (benötigt Position der Knochen und Gewicht)
  \begin{itemize}
   \item Drehmoment $\vec{M} = \vec{r} \times \vec{F}$, mit Ortsvektor $\vec{r}$ vom Bezugspunkt des Drehmoments zum Angriffspunkt der Kraft $\vec{F}$. Betrag des Drehmoments $M = r \cdot F \cdot \text{sin}(\alpha)$, mit dem Winkel $\alpha$ zwischen $\vec{F}$ und $\vec{r}$.\\
   (\url{https://de.wikipedia.org/wiki/Drehmoment})
   
   \item Kraft, die auf Knochen wirkt: resultierende Kraft der Streckenlast, die durch zugewiesenes Gewicht erzeugt wird. Der Einfachheit halber einfach rechteckige Streckenlast annehmen, dann greift Kraft am Mittelpunkt der Strecke und $\vec{F} = \frac{1}{2} m g$\footnote{Die Formel zu rechteckigen Streckenlasten, die man so findet ist $\vec{F} = \frac{1}{2} L q_0$, mit Länge der Strecke $L$ und den auf der Strecke verteilten Einzellasten $q_0$, aber $L q_0 = m g$. Daraus folgt obige Formel.}, mit Masse $m$ und Erdbeschleunigung $g$\\ (\url{https://www.der-wirtschaftsingenieur.de/index.php/streckenlast/}, \url{https://www.youtube.com/watch?v=T1Sf8GPD4dc})
   
   \item Reduzierung des Problems auf zwei Dimensionen, da Körper so aufgebaut ist, dass er nur nach vorne oder hinten umkippen kann und nicht zur Seite.
   
   \item Damit Körper immer in Balance ist, muss auch nicht terminalen Teilen ein Drehmoment (und damit auch implizit ein Gewicht) zugewiesen werden. Werden aus diesen Nichtterminalen dann neue Teile generiert, so muss das Drehmoment bzw.\ die Summe der einzelnen Drehmomente gleich bleiben.
   
   \item Der Bezugspunkt für die Berechnung der Drehmomente ist der Schwerpunkt des Körpers, da sich dort alle durch das Eigengewicht wirkenden Drehmomente des Körpers zu null aufaddieren.\\ (\url{https://www.grund-wissen.de/physik/mechanik/drehmoment-und-gleichgewicht.html})\\
   Das heißt zu Beginn der Generierung muss der Schwerpunkt des Körpers festgelegt werden, damit die Drehmomente berechnet werden können. Das ist am Anfang aber sehr einfach, da das initiale Element die Bounding Box des Gesamten Skeletts ist. Also ist der Schwerpunkt der Mittelpunkt.
  \end{itemize}

 \item Lage der Wirbelsäule am Anfang festlegen (?), da sie der zentrale Teil des Körpers ist und die Lage der anderen Körperteile grob festlegt. Form variiert von Gerade (Fisch) bis geschwungener Hals und Schwanz (z.B. Giraffe). Mensch passt hier nicht ganz ins Schema $\rightarrow$ weglassen?
 \item Kombination von Wirbelsäule und Kontaktpunkte mit dem Boden ergeben Lage der Beine.
\end{itemize}


